\chapter{Evaluation}


\section{Evaluation Plan}

\subsection{Optimization Problems Definition}
\subsubsection{Traveling Salesman Problem}
\paragraph{tsplib95 benchmark set}
which problem I want to solve with hyper-heuristic
$http://comopt.ifi.uni-heidelberg.de/software/TSPLIB95/STSP.html$

\subsection{Hyper-Heuristic Settings}
To evaluate the performance of developed system we first need to compare it with the base line. In our case it is the simple meta-heuristic that is solving the problem with static hyper-parameters.

In order to organize the evaluation plan, we distinguish two stages of setup, where different approaches could be applied. 
At the first stage we select Low Level Heuristic, while at the second one we select hyper-parameters for LLH.
The approaches for each step are represented in table \ref{evaluation: settings table}.

% highlight that such MHs as Simulated Annealing are now "restarting". It measn, that is in SA case, the Temperature parameter change drops to initial state. Thus here we obtain Iterative Simulated Annealing

\begin{table}[h!]
	\centering
	\begin{tabular}{|l|l|}
		\hline
		\textbf{Low Level Heuristics selection} & \textbf{LLH Hyper-parameters selection} \\
		\hline
		1. Random & 1. Default \\
		2. Multi Armed Bandit & 2. Tuned beforehand \\
		3. Sklearn Bayesian Optimization & 3. Random \\
		4. Static selection of SA, GA, ES & 4. Tree Parzen Estimator \\
		& 5. Sklearn Bayesian Optimization\\
		\hline
	\end{tabular}
	
	\caption{System settings for benchmark}
	\label{evaluation: settings table}
\end{table}


For instance, mentioned above baseline could be described as $Settings 4.1.$ for meta-heuristics with default hyper-parameters and as $Settings 4.2.$ for meta-heuristics with tuned beforehand hyper-parameters.

For our benchmark we selected following settings sets:

\begin{itemize}
	\item \textit{Baseline:} $4.1, 4.2;$
	\item \textit{Random Hyper-heuristic:} $1.1, 1.2, 1.3, 4.3;$
	\item \textit{Parameter control:} $4.4, 4.5;$
	\item \textit{Selection Hyper-Heuristic:} $2.1, 3.1, 2.2, 3.2;$
	\item \textit{Selection Hyper-Heuristic with Parameter Control:} $2.4, 2.5, 3.4, 3.5;$
\end{itemize}

Each of this settings will be discussed in details in following section.


\subsection{Selected for Evaluation Hyper-Heuristic Settings}
\paragraph{Baseline}

\paragraph{Hyper-heuristic With Random Switching of Low Level Heuristics}

\paragraph{Parameter control}

\paragraph{Selection Only Hyper-Heuristic}

\paragraph{Selection Hyper-Heuristic with Parameter Control}


\section{Results Discussion}

\subsection{Baseline Evaluation}

\paragraph{Meta-Heuristics With Default Hyper-Parameters}

\paragraph{Meta-Heuristics With Tuned Hyper-Parameters}

\paragraph{Results Description and Explanation}


\subsection{Hyper-Heuristic With Random Switching of Low Level Heuristics}

\paragraph{Results Description and Explanation}


\subsection{Parameter Control}

\paragraph{Results Description and Explanation}


\subsection{Selection Only Hyper-Heuristic}
auto-sklearn paper, p.2 - comparison of GP and TPE BOs.

\paragraph{Results Description and Explanation}


\subsection{Selection Hyper-Heuristic with Parameter Control}

\paragraph{Results Description and Explanation}

\section{Conclusion}
