\chapter{Introduction}\label{intro}
\todor{do it in goal-problem-solution style}
\todor{research questions are missing}

\section{Motivation}
- one MH could be better than other for particular problem and user could not predict it \cite{surv:kerschke2019automated}
- no-free-lunch (NFL) theorem should be discussed


\section{Objectives}
- define a search space of MHs - dynamic meta-heuristics selection
- support parameter tuning on the fly
- create / find portfolio of MHs

  	
\section{Overview}
The description of this thesis is organized as follows: In Chapter 2, we extend not advanced in field of problem solving and optimizations reader by background knowledge and defines scope of thesis. Chapter 3 describes related work in defined scope. In Chapter 4 one will find the concept description of dynamic heuristics selection. Chapter 5 contains more detailed information about approach implementation and  embedding it to BRISE. The evaluation results and analysis could be found in Chapter 6. Finally, Chapter 7 concludes the thesis and Chapter 8 describe the future work.
