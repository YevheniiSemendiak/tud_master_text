\chapter{Introduction}\label{intro}
\paragraph{Intent and content of chapter.} This chapter is an self-descriptive, shorten version of thesis.



\section{Motivation}
% goal-problem-solution style: definition and motivation of the goal as following paragraphs
Structure:
\begin{itemize}
	\item optimization problem(OP) $\rightarrow$ exact or approximate (+description to both) $\rightarrow$ motivation to use \textbf{approximate solvers} $\rightarrow$
	\item impact of parameters, their tuning on solvers $\rightarrow$ motivation of \textbf{parameter control} (for on-line solver) $\rightarrow$
	\item but what if we want to solve a class of problems (CoP) $\rightarrow$ algorithms performance is different $\rightarrow$ 
	\item user could not determine it \cite{surv:kerschke2019automated} $\rightarrow$ exploration-exploitation balance
	\item no-free-lunch (NFL) theorem \cite{wolpert1997no} $\rightarrow$ motivation of the thesis
\end{itemize}

\paragraph{thesis motivation} The most related research field is Hyper-heuristics optimizations \cite{burke2003hyper}, that are designed to intelligently choose the right low lewel heuristics (LLH) while solving the problem.
But the weak side of hyper-heuristics is the luck of parameter tuning of those LLHs [links].
In the other hand, meta-heuristics often utilize parameter control approaches [links], but they do not select among underlying LLHs.
The goal of this thesis is to get the best of both worlds - algorithm selection from the hyper-heuristics and parameter control from the meta-heuristics.


\section{Research objective}
\todoy{Rename: Problem definition?}
% goal-problem-solution style: problem clarification, research questions
The following steps should be completed in order to reach the desired goal:

\paragraph{Analysis of existing studies of algorithm selection.} \textit{(find a problem definition, maybe this will do \cite{surv:kerschke2019automated})}

\paragraph{Analysis of existing studies in field of parameter control and algorithm configuration problems} \textit{(find a problem definition)} \cite{lavesson2006quantifying}

\paragraph{Formulation and development of combined approach for LLH selection and parameter control.}

\paragraph{Evaluation of the developed approach with \todoy{family of problems??? since it is a HH, maybe we should think about it...}.}

\paragraph{Research Questions} At this point we define a Research Questions (RQ) of the Master thesis.
% https://writingcenter.fas.harvard.edu/pages/beginning-academic-essay - "Focus the Essay" part.

\begin{itemize}
	\item \textbf{RQ 1} Is it possible to select an algorithm and it hyper-parameters while solving an optimization problem \textit{on-line}?

	\item \textbf{RQ 2} What is the gain of selecting and tuning algorithm while solving an optimization problem?

	\item \textbf{RQ 3?} How to solve the problem of algorithm selection and configuration simultaneously?
	
\end{itemize}


\section{Solution overview}
% goal-problem-solution style: here goes a solution description
\todoy{Rename: Problem solution?}

\begin{itemize}
	\item described problems solved by HH, highlight problems of existing HHs(off-line, solving a set of homogeneous problems in parallel)
	\item create / find portfolio of MHs (Low level Heuristics)
	\item define a search space as combination of LLH and their hyper-parameters (highlight as a contribution)
	\item solve a problem on-line selecting LLH and tuning hyper-parameters on the fly. (highlight as a contribution? need to analyze it.)
\end{itemize}

% i guess, it will be changed, no need to re-write it all the time :D
\paragraph{Thesis structure}
The description of this thesis is organized as follows. First, in chapter \ref{chap:background} we refresh readers background knowledge in the field of problem solving and heuristics. In this chapter we also define the scope of thesis. Afterwards, in chapter \ref{chap:background} we describe the related work and existing systems in defined scope. In Chapter 4 one will find the concept description of dynamic heuristics selection. Chapter 5 contains more detailed information about approach implementation and  embedding it to BRISE. The evaluation results and analysis could be found in Chapter 6. Finally, Chapter 7 concludes the thesis and Chapter 8 describe the future work.

