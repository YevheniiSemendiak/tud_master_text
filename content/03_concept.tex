\chapter{Concept description}
\paragraph{In this chapter} we describe the concept of developed selection Hyper-Heuristic with parameter control, not diving deep into the implementation details.
The best practice in software engineering is to minimize an effort for the implementation and reuse already existing, well-tested and broadly used code.
With this idea in mind we had decided to use one of previously created (and highlighted by us in \ref{bg: parameter tuning}) hyper-parameter tuning systems as the code basis and those turn it into the core of hyper-heuristic.
We also reuse the set of already developed heuristics as the Low Level Heuristics for designed Hyper-Heuristic.

The structure of this chapter is as follows.
First, we define the Search Space entity in \ref{concept:search space}, it requirements and structure. 
It should bound the world of Low Level Heuristics and the world of Hyper-parameters of those heuristics.


Second, we describe the Prediction process within the previously defined Search Space in \ref{concept:prediction}.
Here we highlight an importance of a prediction model decoupling from the previously defined Search Space structure.
Doing so, we provide certain level of flexibility for user in the usage of different prediction models or developing his own.


Third, in \ref{concept:llh} we gather our attention onto the Low Level Heuristics - a working horse of the hyper-heuristic.
Here we highlight the requirements for LLH in terms of features that will be used by HH.


Later, we select the code basis. We analyze the existing systems and highlight important non-functional characteristics for \ref{concept:code basis selection} hyper-heuristic and \ref{concept:llh code basis selection} low level heuristics set.


Finally, in \ref{concept:changes analysis} we conclude this chapter with the analysis of required changes, that are needed to be accomplished for turning code base system into the hyper-heuristic.


\section{Search Space}\label{concept:search space}
\paragraph{Importance explanation}
\paragraph{Required structure} feature-tree structured



\section{Prediction process}\label{concept:prediction}
\paragraph{Importance explanation}
\paragraph{Requirements} generality, top-down approach of optimization
-- different views of same Configuration (level-dependent) - filtering, transformation
-- consider problem features? while selecting meta-heuristic \cite{surv:kerschke2019automated} page 6


\section{Low Level Heuristics}\label{concept:llh}
\paragraph{Importance explanation}
\paragraph{Requirements}



\section{Code basis selection}
With the aim of effort reuse, the code base should be selected for implementation of the designed hyper-heuristic approach.

\subsection{Hyper-Heuristics Code Base}\label{concept:hh code basis selection}
A.k.a. "brain". Need to find a better way to call this part of HH..
\subsubsection{Requirements}
\subsubsection{Parameter tuning frameworks}
\paragraph{SMAC}
\paragraph{BOHB}
\paragraph{IRACE}
\paragraph{BRISEv2}
\todoy{Maybe, smth else..}
\subsubsection{Conclusion}
BRISEv2 is the best system for code basis.

\subsection{Low Level Heuristics}\label{concept:llh code basis selection}
\subsubsection{Requirements}
\subsubsection{Heuristic frameworks}
-table and short comparison of checked repositories 
% https://docs.google.com/spreadsheets/d/19xjL_ire0R5VLP9seCE5_4sWorSMZP4xvgx7d1q4a9s/edit#gid=0
\paragraph{SOLID}
\paragraph{MLRose}
\paragraph{OR-tools}
\paragraph{pyTSP}
\paragraph{LocalSolver}
\paragraph{jMetalPy}

\subsubsection{Conclusion}
jMetalPy is cool!


\section{Scope of Required Changes}\label{concept:changes analysis}

\subsection{Search Space} highlight - should be done in feature-tree structured search space
\subsubsection{Current state description} What is the problem with the current Search Space?
\subsubsection{Scope of work analysis:} throw away and write a new one :D

\subsection{Prediction Process} highlight - should be done in feature-tree structured search space
\subsubsection{Description of current state}
\subsubsection{Heterogeneous data?} short description of data preprocessing
\subsubsection{Scope of work analysis} 

\subsection{Low Level Heuristics}
\subsubsection{Description of current state}
\subsubsection{Scope of work analysis}

\section{Conclusion of concept}
we selected BRISEv2, jMetalPy because they are cool.
the amount of work is vast so lets dive into it in the next chapter!
