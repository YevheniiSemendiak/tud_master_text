\chapter{Concept Description}
\paragraph{In this chapter} we describe the concept of developed selection Hyper-Heuristic with parameter control, not diving deep into the implementation details.

The structure of this chapter is as follows.
\todoy{maybe we should not highlight the structure of such a small chapter}
First, in in section \ref{concept:search space} we define the Search Space entity requirements and structure. 
It should bound the world of Low Level Heuristics and the world of Hyper-parameters of those heuristics.


Next, we describe the Prediction process within the previously defined Search Space in section \ref{concept:prediction}.
Here we highlight an importance of a prediction model decoupling from the previously defined Search Space structure.
Doing so, we provide certain level of flexibility for user in the usage of different prediction models or developing his own.


Finally, in section \ref{concept:llh} we gather our attention onto the Low Level Heuristics - a working horse of the hyper-heuristic.
Here we highlight the requirements for LLH in terms of features that will be used by HH.


\section{Search Space}\label{concept:search space}
\paragraph{Importance explanation}
\paragraph{Required structure} feature-tree structured



\section{Prediction Process}\label{concept:prediction}
\paragraph{Importance explanation}
\paragraph{Requirements} generality, top-down approach of optimization
-- different views of same Configuration (level-dependent) - filtering, transformation
-- consider problem features? while selecting meta-heuristic \cite{surv:kerschke2019automated} page 6


\section{Low Level Heuristics}\label{concept:llh}
\paragraph{Importance explanation}
\paragraph{Requirements}


\section{Conclusion of concept}
to be done...