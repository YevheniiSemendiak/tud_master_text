\documentclass[ms,english]{stthesis}

% --- Load packages START
\usepackage[T1]{fontenc}
\usepackage{libertinus}

\usepackage[utf8]{inputenc}
\usepackage[english]{babel}
\usepackage{isodate}

% Bibliography
\usepackage[
    style=numeric-comp,
    backend=biber,
    url=false,
    doi=false,
    isbn=false,
    hyperref,
]{biblatex}

\addbibresource{content/bibliography.bib}

\usepackage{hyperref} % makes all links clickable but hides ugly boxes
\usepackage[capitalise,nameinlink,noabbrev]{cleveref} % automatically inserts Fig. X in the text with \cref{..}
\usepackage{todonotes}
\usepackage{pdfpages} % to insert pdf pages
\usepackage{xcolor}
\usepackage{csquotes}

% insert python code snipets
\usepackage{listings}
\definecolor{codegreen}{rgb}{0,0.6,0}
\definecolor{codegray}{rgb}{0.5,0.5,0.5}
\definecolor{codepurple}{rgb}{0.58,0,0.82}
\definecolor{backcolour}{rgb}{0.95,0.95,0.92}

\lstdefinestyle{mystyle}{
	backgroundcolor=\color{backcolour},   
	commentstyle=\color{codegreen},
	keywordstyle=\color{magenta},
	numberstyle=\tiny\color{codegray},
	stringstyle=\color{codepurple},
	basicstyle=\ttfamily\footnotesize,
	breakatwhitespace=false,         
	breaklines=true,                 
	captionpos=b,                    
	keepspaces=true,                 
	numbers=left,                    
	numbersep=5pt,                  
	showspaces=false,                
	showstringspaces=false,
	showtabs=false,                  
	tabsize=2
}

\lstset{style=mystyle} % enable cref for listing
\crefname{lstlisting}{Listing}{listings}
\Crefname{lstlisting}{Listing}{Listings}

\lstnewenvironment{code}[1][]% use 'code' instead of 'lstlisting' to prevent splitting between pages
{
	\noindent
	\minipage{\linewidth} 
	\vspace{0.5\baselineskip}
	\lstset{basicstyle=\ttfamily\footnotesize,#1}}
{\endminipage}

\usepackage{caption}
\usepackage{subcaption}
\captionsetup{font=sf,labelfont=bf,labelsep=space}
\usepackage[section]{placeins} % appendix figures alignment
\usepackage{floatrow}
\floatsetup{font=sf}
\floatsetup[table]{style=plaintop}
\captionsetup{singlelinecheck=off,format=plain,justification=centering}
\DeclareCaptionSubType[alph]{figure}
\DeclareCaptionSubType[alph]{table}
\captionsetup[subfloat]{labelformat=brace,list=off}
\usepackage{multirow}
\usepackage{wrapfig}

\usepackage{booktabs} % fancy tables. need to analyse.
\usepackage{array}
\usepackage{tabularx}
\usepackage{tabulary}
\usepackage{tabu}
% \usepackage{longtable} % tables for the number of pages
%\usepackage{quoting}
\usepackage[babel]{microtype}
\usepackage{xfrac}
\usepackage{enumitem}
\usepackage[perpage]{footmisc} % reset footnote counter on each page
\usepackage{epigraph}

\usepackage{ellipsis}
\let\ellipsispunctuation\relax

\usepackage{svg}
% --- Load packages END

\newcommand{\todor}[1]{\todo[color=green,inline,size=\small]{Reviewer: #1}}
\newcommand{\todoy}[1]{\todo[color=yellow,inline,size=\small]{Yevhenii: #1}}

% usefull links
% https://writingcenter.fas.harvard.edu/pages/developing-thesis
% https://writingcenter.fas.harvard.edu/pages/beginning-academic-essay 
% by defining a scope we narrow a topic. need to define it carefully, I think, the one we had in pre-defence will be slightly changed

\begin{document}
  \includepdf{static/thesis_title.pdf}
  
  \includepdf{static/taskDescription.pdf}
  
  \newpage
\section*{Abstract}
The importance of balance between exploration and exploitation plays a crucial role while solving combinatorial optimization problems. This balance is reached by two general techniques: by using an appropriate problem solver and by setting its proper parameters. Both problems were widely studied in the past and the research process continues up until now. The latest studies in the field of automated machine learning propose merging both problems, solving them at design time and later strengthening the results at runtime. To the best of our knowledge, the \emph{generalized} approach for solving the parameter setting problem in heuristic solvers has not yet been proposed. Therefore, the concept of merging heuristic selection and parameter control has not been introduced.

In this thesis we propose an approach for generic parameter control in meta-heuristics by means of reinforcement learning (RL). Making a step further, we suggest a technique for merging the heuristic selection and parameter control problems and solving them at runtime using RL-based hyper-heuristic. The evaluation of the proposed parameter control technique on a symmetric traveling salesman problem (TSP) revealed its applicability by reaching the performance of tuned in offline and used in isolation underlying meta-heuristic. Our approach provides the results on par with the best underlying heuristics with tuned parameters.
  \tableofcontents
  \chapter{Introduction}\label{intro}

\section{Motivation}
Traditionally, optimization problem (OP) in general and combinatorial OP in particular may be tackled by a number of different approaches. Besides an exhaustive brute-force search (comparison of all possible solutions) a more convenient methodologies do exist. Most of the frequently used approaches can be grouped into three large families~\cite{junger2003combinatorial,biegler2004retrospective,festa2014brief}: \emph{exact solvers}, \emph{approximate solvers} and \emph{heuristics}. The first two among them are build on mathematical proofs and provide a guarantee on the solution quality, while the later is often based on the usage of domain knowledge and stochastic processes to guide the search. When it comes to heuristic solvers, their performance is defined by the provided strength in exploration and exploitation (EvE) and in the exposed balance between them. 

The EvE balance is controlled on a several levels. Firstly, the majority of algorithms is configurable by means of their exposed parameters, which are also often called \emph{hyper-parameters}. Their proper selection is also an OP and named \emph{parameter settings problem} (PSP). Solving this OP often requires a considerable effort and has a great influence on heuristic performance~\cite{lavesson2006quantifying}. The resolution may be done \emph{before} running the algorithm (design time), or \emph{while} it solves the OP (runtime). The former approach is also called \emph{parameter tuning}, for which numerous systems were created to solve the problem on a generic level~\cite{hutter2009paramils,hutter2011sequential,lopez2016irace,falkner2018bohb,brise2spl}. A key assumption of the parameter tuning systems is an expansiveness of target system evaluation, therefore, they build surrogate models to simulate direct evaluations. The later approach, namely, \emph{parameter control} originally was introduced by the research community of evolutionary algorithms~\cite{karafotias2014parameter} and nowadays is spread to other heuristic solvers, but in an \emph{algorithm-dependent} manner. Unfortunately, even a proper parameter setting may not lead to the best performing algorithm for problem at hand. This issue was formalized by the \emph{no-free-lunch theorem for optimization} (NFLT)~\cite{wolpert1997no}, which states that ``all search algorithms have the same performance, when their results are averaged over all possible optimization problems''. Therefore, the second approach for reaching EvE balance, which resolves the consequence of NFLT is the problem of a proper algorithm selection (ASP). The commonly used approach for ASP solving is the hyper-heuristics utilization. Depending on the learning time, they may perform low-level heuristic selection at the design- or runtime~\cite{burke2019classification}. One among commonly used approaches for on-line selection hyper-heuristic is the usage of reinforcement learning (RL) approaches~\cite{moriarty1999evolutionary,mcclymont2011markov}, while in the design time a regular parameter tuning techniques could be used.

The research does not stand still and nowadays the attempts to merge ASP and PSP problems are actively making. Concretely, in an \emph{automatic machine learning} field, such problem was formalized as the \emph{combined algorithm selection and parameter setting} problem (CASH)~\cite{thornton2013auto}. Several frameworks were created to resolve ML CASH problem~\cite{thornton2013auto,feurer2015efficient,olson2019tpot}. The majority of them are based on the existing parameter tuning systems, but those frameworks are (1) purely related to ML field and (2) acting at the design time due to ML techniques specifics. As for heuristics, an explicit studies for ASP and PSP merging in runtime were not found. One may follow the ML approach of joining ASP and PSP into united search space and utilizing the same parameter tuning systems. However, an important characteristic of such spaces appears that prohibits treating them as a \emph{flat} structure: they comprise all underlying algorithms hyper-parameters. Since one solver type requires only its specific parameters and prohibits (or ignores) the parameters of other algorithms, the resulting configuration space will happen to be `sparse'. In the reviewed systems aforementioned dependencies are treated properly and surrogate models are able, with help of some tricks, learn them. But, the proposed surrogate optimization techniques in an extremely sparse spaces will straggle while searching a good quality configuration by means of sampling performance. Moreover, the parameter adaptation at the runtime were not introduced on a generic level yet. 


\section{Research objective}\label{intro: research objective}
At this point, we would like to define the \emph{goal of this thesis}. Basing on the existing parameter tuning software, use it to (1) perform the parameter control in meta-heuristics on the generic level and (2) tackle both ASP and PSP while solving the optimization problem at hand. In other words, to propose an \emph{on-line} approach for solving \emph{both} PSP and ASP problems for heuristic solvers.

In order to fulfill the defined above goal, the following \textbf{research questions} should be answered:
\begin{itemize}
	\item \textbf{RQ 1} Is it possible to perform the algorithm configuration at runtime on a generic level?
	
	\item \textbf{RQ 2} Is it possible to simultaneously perform algorithm selection and parameters adaptation while solving an optimization problem?
	
	\item \textbf{RQ 3} What is the effect of selecting and adapting algorithm while solving an optimization problem?
\end{itemize}


\section{Solution overview}
In this thesis we propose the unification of both ASP and PSP into a single problem. For doing so, we firstly introduce a generic runtime PSP solution; secondly we propose joining several PSPs search spaces into united APSP on an algorithm type.

To overcome the sparseness issue we propose a complex solution, which is spread in both search space structure and prediction process. For the APSP representation we propose the usage of data structure, similar to feature trees from software product lines field. Doing so we treat solver type and its hyper-parameters uniformly as a regular parameter, however of the different types (categorical vs numerical). The dependencies between parameters in such search space are explicitly handled in form of parent-child relationship. As a result, the search space could be viewed as a layered structure, where on the first level remain (categorical) parameter defining algorithm type, and on the level(s) below a respective hyper-parameters (categorical and numerical). The prediction process is made level-by-level. Thus, in the united APSP we firstly build a surrogate model for the algorithm type prediction. Afterwards, when the solver type is selected, we filter available information to operate on only relevant to the selected algorithm type data. With this filtered data we build a surrogate for the second level and predict the parameters for selected algorithm. The dependencies among algorithm-related parameters are also treated in form of parent-child relationship, therefore, we proceed the level-wise prediction proceeds until obtaining a completed configuration. In our work we utilize RL techniques solve the underlying OP iteratively on-line. On each step, according to available performance evidences for current optimization session, we select among any-time meta-heuristics (portfolio of algorithms), predict its best-performing hyper-parameters all by means of created for each layer surrogate models. After constructing such configuration, we continue solving the underlying OP to obtain new evidences for the next iteration.

The structure of this thesis is organized as follows. Firstly, in \cref{bg} we refresh the reader background knowledge in the field of optimization problems, solver types focusing on heuristics. We also review the parameter setting and the available solutions for this problem. In \cref{Concept description} one will find the description of the proposed approach for generic parameter control and APSP problem unification. The structural and functional requirements for the search space representation and the sampling process are also located there. \cref{impl} is dedicated to the review of implementation details, among which a code basis selection, aforementioned requirements realization and a work-flow representation. The evaluation results of the proposed concept and their discussion could be found in \cref{eval}. \cref{conclusion} concludes the thesis and \cref{future work} describes the future work.

  \chapter{Background and Related Work Analysis}\label{bg}
In this chapter we provide reader with the base knowledge in field of Optimization Problems and the process of their solving.
The reader who is an expert in field of Optimization and Search Problems could find this chapter as an obvious discussion of well-known facts. If the notions of \textit{Parameter Tuning} and \textit{Parameter Control} seems like two different names for one thing, we encourage you to read this chapter carefully.
We highly recommend for everyone to refresh the knowledge of sections topics and examine the examples of Hyper-Heuristics in \ref{bg: hh examples} and Systems for parameter tuning in \ref{bg: parameter tuning expamples} since we use them later in concept implementation.

\paragraph{In this chapter we...} 


\section{Optimization, Search Problems and their Solvers}\label{bg:opt problems and solvers}


\subsection{Optimization Problems and Their Characteristics}
%https://www.solver.com/problem-types
% need to add some kind of catchy intro, here or in previous parts of this section.
While the Search Problem (SP) defines the process of finding a possible Solution for the Computation Problem, an Optimization Problem (OP) is the special case of the SP, focused on the process of finding the 'best possible' Solution for Computation Problem~\cite{goldreich2010p}. 


In this thesis we focus on the Optimization Problems — a special case of the Search Problems.


A lot of conducted studies in this field have tried to formalize the concept of OP, but the underlying notion such a vast that it is almost impossible to exclude the application domain from the definition. The description of every possible Optimization Problem and all approaches for solving it are out of the scope of this thesis. However, a birds-eye view should be presented in order to make sure that reader is familiar with all notions used through this thesis. 


In \cite{biegler2004retrospective,figueira2014hybrid,amaran2016simulation} authors distinguished OP characteristics that overlap through each of these works and those we would like to start from them.


First, let us define the subject of the Optimization. In general, it could be imagined as the Target System (TS) displayed on picture \ref{bg:pic:Target System}. Analytically it could be represented as the function $Y = f(X)$. Informally it accepts the information with its \textit{inputs} \textbf{X} sometimes also called variables or parameters, performs a \textit{Task} and produces the result on its \textit{outputs} \textbf{Y}.

\svgpath{{graphics/Background/}}
\begin{figure}
	\centering
	\includesvg[width=0.5\textwidth]{TargetSystem}
	\caption{Target System}
	\label{bg:pic:Target System}
\end{figure}

Pair of $X$ and respective $Y$ form a \textit{Solution} for Computation Problem.
All possible inputs $X$ form a \textit{Search Space}, while all outcomes $Y$ form an \textit{Objective Space}.
The Solution could also be characterized by the \textit{objective} value(s) — a quantitative measure of TS performance that we minimize or maximize. 
We could obtain those value(s) directly by reading the $Y$, or indirectly for instance, noting the time TS took to produce the output $Y$ for given $X$. 
The Solution objective value(s) form object(s) of Optimization. 
For the sake of simplicity we here use $Y$, \textit{outputs}, \textit{objectives} and $X$, variables, \textbf{parameters(?)}% will decide later
 interchangeably.


Next, let us highlight the Target System characteristics.
Among mentioned in \cite{biegler2004retrospective,figueira2014hybrid,amaran2016simulation} we found those the most important:
\begin{itemize}[itemsep=8pt]
	\item \textbf{Input data types} of $X$ is a crucial characteristic. The variables could be either \textit{discrete} where representatives are binary strings, integer-ordered or categorical data, % One could apply mixed integer linear (nonlinear) programming here (MILP, MINLP) \cite{biegler2004retrospective}.
	\textit{continuous} where variables are usually a range of real numbers, or \textit{mixed} as the mixture of previous two cases.

	\item \textbf{Constrains} are functional dependencies that describe the relationships among inputs and defile the allowable values for them.

	\item \textbf{Amount of knowledge} TS exposes about the dependencies between $X \rightarrow Y$ or objective values. With respect to this knowledge, the Optimization could be \textit{White Box} — the TS exposes it internals fully, so it is even possible to derive the algebraic model of TS.
	%\textit{Gray Box} - the amount of exposed knowledge is significant, but not enough to build the algebraic model.
	\textit{Black Box} — the exposed knowledge is mostly negligible.
	%In this case the Derivative Free Optimization approaches (such as Surrogate Optimization, different Meta-|Hybrid-|Hyper-Heuristics)  are applicable.

	%\paragraph{Dependency types} could be  The inputs to outputs dependencies the of Target System could also be distinguished form perspective of linearity \cite{biegler2004retrospective,figueira2014hybrid}.
	%\textit{Linear dependencies} reveal the Linear Programming Optimization approaches, while with \textbf{Nonlinear dependencies} one should consider Nonlinear Programming.

	\item \textbf{Dependencies randomness} One of possible challenges, while obtaining the knowledge about TS is uncertainty of output. Ideal case is the \textit{deterministic} dependency between $X$ and $Y$, however in most of real-world challenges engineers tackle with the \textit{stochastic} systems whose output is affected by random processes. 

	\item \textbf{Cost of evaluation} is the amount of resources (computational, time, money, etc.) TS will spend to obtain the result for particular input. It varies from very cheap if the TS is a simple algebraic formula and Task is to evaluate it, to very expensive if the TS is a complex Neuron Network and the Task is to train it on data.

	\item \textbf{Number of objectives} could be either \textit{Single}, or \textit{Multiple}. According to the number of objectives, the result of optimization will be either single Solution, or set of non-dominated (Pareto-optimal) Solutions \cite{deb2014multi}.

\end{itemize}


Combining different characteristic, one could obtain broad range of Optimization Problem types.


In this thesis we tackle such a real life problems as bin packing, job-shop scheduling or vehicle routing.
The mentioned above problems has been shown to be NP-complete computational complexity \cite{garey1979computers}.


As an example, let's grasp these characteristics for Traveling Salesman Problem (TSP) \cite{applegate2006traveling} — an instance of vehicle routing problem and one of the most studied combinatorial OP, yet still remaining one of the most challenging (here we consider deterministic, symmetric TSP).
The informal definition of TSP is as follows: 'Given a set of cities and the distances between each of them, what is the shortest path to visit each city once and return to the origin city?'.
The input data (path) is a vector of city indexes, and those the type is a non-negative integers \textit{0, 1, 2...}.
There are two constrains on path: it should contain only unique indexes (those, each city will be visited only once) and it should start and end from the same city. 
The TSP distance (or cost) matrix here plays role of Target System, clearly that this TS exposes all internal knowledge and those it is the white box.
Since the cost matrix is fixed and not changing, the TS is considered to be deterministic, cost for two identical paths are always the same (although there exist Dynamic TSP where the cost matrix changes while computing the path cost to reflect a real-time traffic information updates while traveling \cite{cheong2011dynamic}).
It is extremely cheap to compute a cost for given path using cost matrix, those overall Solution evaluation in this TS is cheap.
Since we are optimizing only the route distance, it is a Single objective OP.


\subsection{Optimization Problem Solvers and Their Classes}
Any Optimization Problem could be solved by an exhaustive search. 
But when the problem size significantly increase, the amount of time needed for an exhaustive search becomes infeasible and in most cases even relatively small problem instances could not be solved by an enumeration.

Here different techniques come into play, but the provided by Target System characteristics of Optimization Problem could restrict and sometimes strictly define the applicable approach.
For instance, imagine you have white box deterministic TS with discrete constrained input data and cheap evaluation. The OP in this case could be described using Integer Linear Programming \todoy{ref} approaches, or heuristics \todoy{ref}. If this TS turned out to be a black box, the ILP approaches are not applicable and one should consider using heuristics \cite{biegler2004retrospective}.


Again, there exist a lot of different facets for OP Solvers classification, however they are a subject of surveying works. Here as the point of interest we decided to highlight two of them.

From the perspective of solution quality:
\begin{itemize}
	\item \textbf{Exact} Solvers are those algorithms that always provide an optimal Solution for OP.
	\item \textbf{Approximate} Solvers produce a sub-optimal output with guarantee in quality (some order of distance to the optimal solution).
	\item \textbf{Heuristics} Solvers do not give any worst-case guarantee for the final result quality.
\end{itemize}

From the perspective of solution availability:
\begin{itemize}
	\item Algorithms that expose the Solution \textbf{at the end} of their run.
	\item In opposite, \textbf{anytime} algorithms designed to improve the solution quality step-by-step while solving the OP and those, intermediate results are naturally accessible. 
\end{itemize}

Each if this algorithm families has its own advantages and disadvantages in comparison to other, and require more detailed description in following chapters.

\subsubsection{Solution Quality Perspective}
\paragraph{Exact Solvers}
As we stated previously, the exact algorithms are those which always solve an OP to optimality.

For some OP it is possible to develop an algorithm that is much faster than exhaustive search — it runs in super-polynomial time providing an optimal solution. As it stated in \cite{woeginger2003exact}, if the common belief $P \ne NP$ is true, those super-polynomial time algorithms are the best we can hope to get when dealing with an NP-complete problem.

By the definition in \cite{fomin2013exact}, the objective of an exact algorithm is to perform better (in terms of running time) than exhaustive search.
In both works \cite{woeginger2003exact} author had enumerated the main techniques for exact algorithms designing each of which enhance this 'better' independently.
A brief explanation of them will help to refresh the knowledge.

\begin{itemize}
	\item \textbf{Branching and bounding} techniques when applied to origin problem, split the search space of all possible solutions (e.g. exhaustive search space) to a set of smaller sub-spaces (more formally, branching the search tree into subtrees). This is done with an intent to later prove that some sub-spaces never lead to an optimal solution and those could be ignored in order to speed-up the search.
	
	\item \textbf{Dynamic programming across the Subsets} techniques in some sort could be combined with the mentioned above branching techniques. After forming the Search Space subsets (branches), the dynamic programming attempts to derive solutions for smaller subsets and combine them into solutions for lager subsets unless finally derive a solution for original search space.
	
	\item \textbf{Problem preprocessing} could be applied as an initial phase of the solving process. This technique is dependable upon the underlying OP, but when applied properly, significantly reduce the running time. A toy example from \cite{woeginger2003exact} elegantly illustrate this technique: imagine problem of finding a pair of two integers $x_i$ and $y_i$ that sum up to integer $S$ in $X_k$ and $Y_k$ sets of unique numbers ($k$ here denotes the size of a set). The exhaustive search will enumerate all $x-y$ pairs in $O(k^2)$ time. But one could first preprocess the data by sorting it, after that use bisection search repeatedly in this sorted array and search for $k$ values $S - y_i$, the overall time complexity becomes $O(k\log(k))$.
\end{itemize}


\paragraph{Approximate Solvers} (or approximate algorithms) as representatives of theoretical computer science, have been created in order to tackle the computationally difficult OP, which is handful when the OP is $NP-hard$. %In words of Garey and Johnson it means "I can't find an efficient (polynomial time) algorithm, but neither can all of these famous people."
If the widely believed conjecture $P \ne NP$ is true, a wide range of OPs cannot be solved with exact solvers efficiently (in polynomial time).


In contradistinction to exact, these algorithms find an approximate solution effectively with the provable assurances on the distance from an optimal solution \cite{williamson2011design}. The worst case results quality guarantee is crucial in design of approximation algorithms and involves mathematical proofs.
A lot of OPs have a \textit{polynomial-time approximation schemes} so the approximate solvers could be applied to them, but not all problems have those schemes. For instance, one could apply approximate solver for TSP or Knapsack Problem, but not for Maximum Satisfiability Problem \cite{williamson2011design}.

\paragraph{Heuristics} in contradiction to approximate solvers do not provide any guarantee on distance of provided result to optimal result.

\subsubsection{Motivation of Heuristics}
An old engineering slogan says, "Fast. Cheap. Reliable. Choose two."
Approximation Solvers are representatives of the theoretical computer science field. 
Pros and cons of both \cite{hromkovivc2013algorithmics}
Indeed, in real life use-cases sometimes it worse to sacrifice the optimal Solution quality in order to obtain near-optimal quality, but much faster.


\section{Heuristic Solvers for Optimization Problems}
TSP as the running example. I guess, I will introduce it as an example of perturbation problems in previous section.\ref{sec:opt problms, solvrs}

\subsection{Heuristics}
\subsubsection{Definition}
\subsubsection{Examples}

\subsection{Meta-Heuristics}
\subsubsection{Definition}
\subsubsection{Classification}
\subsubsection{Examples}
We distinguish following examples among all existing meta-heuristics, since later we use them as the LLH in developed hyper-heuristic.
\paragraph{GA}
\paragraph{SA}
\paragraph{ES}

\subsection{Hybrid-Heuristics}
\subsubsection{Definition}
\subsubsection{Examples}
\paragraph{Guided Loca Search (GLS) + Fast Local Search} \cite{tsang1997fast}
\paragraph{Direct Global + Local search} \cite{syrjakow1999efficient}
\paragraph{Simulated Annealing + Local Search} \cite{martin1996combining}

\subsubsection{No-Free-Lunch Theorem}
NFL is the problem of heuristics\cite{wolpert1997no}
\subsubsection{Exploration-Exploitation Balance}
\subsubsection{Conclusion} 
Proper assignment of hyper-parameters has great impact on exploration-exploitation balance and those on (meta)~-heuristic performance. 

\subsection{Hyper-Heuristics}
\subsubsection{Definition}
\subsubsection{Classification}
\paragraph{Search Space:} heuristic selection, heuristic generation
\paragraph{Learning time:} on-line learning hyper-heuristics, off-line learning hyper-heuristics, no-learning hyper-heuristics
\paragraph{Other classification characteristics} from \cite{surv:kerschke2019automated}, \cite{burke2019classification}, mb smth else. For instance, hyperparameter tuning
\subsubsection{Examples}\label{bg: hh examples}% should I present it in following sections?
\cite{surv:drake2019recent} (Online algorithm selection at page 27); \cite{surv:kerschke2019automated}

\subsection{Conclusion on Approximate Solvers}
\paragraph{Pros and cons of heuristics} - Heuristics are strictly problem dependent and each time require adaptations.
\paragraph{Pros and cons of meta-heuristics} - no LLH selection, strict to one problem
\paragraph{Pros and cons of hybrid-heuristics} - no LLH selection, strict to one problem ? 
\paragraph{Pros and cons of hyper-heuristics} - no parameter control?


\section{Parameter Tuning as a Search Problem}\label{bg: parameter tuning}
The goal of section: analysis of existing systems for hyper-parameter optimization (tuning), weaknesses and strength of each of the system

\subsection{Parameter Tuning Problem Definition}
\subsection{Approaches for Parameter Tuning}
\paragraph{Grid Search}
\paragraph{Random Search}
\paragraph{Model Based Search}

\subsection{Systems for Model Based Parameter Tuning}\label{bg: parameter tuning expamples}

\subsubsection{IRACE}
\paragraph{approach} \cite{irace:lopez2016irace}
\paragraph{pros and cons}

\subsubsection{SMAC}
\paragraph{approach description}

\subsubsection{BOHB}
\paragraph{approach description}

\subsubsection{AUTO-SKLEARN}
\paragraph{CASH (Combined Algorithm Selection and Hyperparameter optimization) problem}
\paragraph{pros and cons (on-line or off-line, problems to solve, extensibility)}\cite{autosklearn:feurer2015efficient}

\subsubsection{BRISEv2}
\paragraph{approach description}
\todoy{Other systems?}


\section{Parameter control as an Optimization Problem}\label{bg: parameter control}
\subsection{Parameter Control Definition}
\subsection{Examples and Reported Impact}
impact of parameter control based on other's evaluation


\section{Conclusion}

The meta-heuristic systems designers reported positive impact of parameter control embedding. 
However, as the outcome of the no-free-lunch theorem, those systems can not tolerate broad range of problems, for instance, problem classes.
In other hand, hyper-heuristics are designed with an aim to select the low level heuristics and those propose a possible solution of problem, stated in no-free-lunch theorem, but the lack of parameter control could dramatically decrease the performance of LLH (probably, I need to find a prove of this, or rephrase).

\paragraph{Scope of thesis defined.} In this thesis we try to achieve the best of both worlds applying the best fitting LLH and tuning it's parameters while solving the problem on-line.

  \chapter{Concept Description}

% TODO: find out what are the other approaches for generic parameter control. by current time I am not able to findout anything generic, except proposed approaches for EAs. but they are for EAs..

Since there exist no universal approach to control the algorithms parameters (Subsection~\ref{bg: parameter control}), our conclusion on the literature analysis was the absence of existing approaches to combine the on-line algorithm selection and the parameter control techniques (Section~\ref{bg: conclusion}). In this Chapter we suggest our methodology to resolve this problem, excluding the implementation details.

In Section~\ref{concept:parameter control}, we suggest the generic parameter control technique and expand the use-case of our solution with algorithm selection. As concluded in the Section~\ref{bg: conclusion}, the main weakness of the reviewed approaches to tackle CASH problems lays in the inability of learning mechanisms to fit and predict in such `sparse' search spaces. The same issue arises in our case, and we resolve it on two levels: (1) in the search space structure and (2) in the prediction process. Firstly, in Section~\ref{concept:search space} we present the joint search space of both algorithm selection and parameter control problems. We outline the functional requirements for such space, followed by the methodology to provide them. Next, we describe the prediction process in Section~\ref{concept:prediction}. Here we highlight an importance of decoupling the learning models from the search space structure. In this way we provide the certain level of flexibility in different learning models usage.

% TODO: check if needed in this chapter
Finally, in Section~\ref{concept: llh} we pay attention onto the low level heuristics (LLH) — a working horses of the hyper-heuristic. Here we highlight the requirements to LLH that are crucial in our case.


\section{Combined Parameter Control and Algorithm Selection Problem}\label{concept:parameter control}
The base idea of the parameter control approaches lays in adapting the solver behavior in the runtime as the response to changes in the solving process (Subsection~\ref{bg: parameter control}). As we mentioned during the heuristics review, the algorithm performance is highly dependent on the provided exploration-exploitation balance (Section~\ref{bg: section heuristics}) which in turn, depends on (1) the algorithm itself and (2) its configuration. The task of parameter control is to optimize the later for gaining the best performance. 

In our work, we tend solve the parameter control problem using the \emph{Reinforcement Learning} (RL) approaches (what is similar to used approaches in EAs~\cite{karafotias2014generic}). % TODO: put it into BG and refer?
The underlying idea of RL could be described as a process of performing actions in some environment with order to maximize the reward obtained after each performed action. To apply this technique onto the parameter control problem, we define what are those \emph{actions} and how to estimate the \emph{reward}. 

Thus, for making the parameter control applicable to broad range of algorithms, we analyze not the solver state itself, but the solution process (in contrast to EAs, where population diversity metrics, etc. are analyzed). To do it, we interrupt~$I$ the solver, analyze~$A$ the intermediate results, set~$S$ the most promising parameters and continue~$C$ solving. The number $i$ of $I \rightarrow A \rightarrow S \rightarrow C$ iterations define the granularity of learning, where one should carefully balance between \emph{time to control} (TTC) the parameters vs \emph{time to solve} (TTS) the problem. Naturally, the limitation of proposed approach is the use-cases, where $TTS >> TTC$.

\todoy{Should I highlight the limitation(s) here or in conclusion and refer from here?}

To evaluate the gained in iteration $i$ reward, instead of using straight solution quality value, we calculate the quality improvement, obtained with the provided configuration $C_i$. Naturally, when the search process converges towards the global optimum, the improvement value tends to decrease, since the amount of significantly better solutions drops. Using the improvement values directly or could confuse the learning models and thus, cause the RL to struggle. To resolve this trouble, the relative improvement (RI) of solution quality is calculated using Formula~\ref{concept: RI formula}, where $S_{i-1}$ and $S_i$ are the solution qualities before and after $i^{th})$ iteration respectively.

The evaluated $C_i \rightarrow RI$ pairs in previous iterations are then used to predict the configuration for next iteration $C_{i+1}$. At this point, we split the sampling process in two steps: (1) hide the search space shape and (2) use the surrogate models for finding configurations that lead to the highest reward.

\todoy{I did not investigate decoupling the surrogate models from the search algorithm to optimize those surrogates (done in Sasha's thesis). Should I mention it somehow, or just postpone and raise the discussion in future work?}


\begin{equation}
RI = \frac{S_{i-1} - S_{i}}{S_{i-1}}
\label{concept: RI formula}
\end{equation}

After obtaining the $C_{i+1}$ configuration, we set it as the solver parameters. To proceed with the solving process, we seed the solver with the solutions from $i-1$ iteration as well.

When it comes to the algorithm selection problem (discussed in the Subsection~\ref{bg: hh}), it turns out that the proposed reinforcement learning approach is also applicable here. We treat the solver type itself as the subject of parameter control and use the proposed RL approach to estimate and use the best performing algorithm, while solving the problem. However, when we add the algorithm type parameter, the resulting search space of become `sparse' and requires special treatment. Two common approaches for tacking such a problem exist. The first requires special kinds of learning-prediction models usage, while the second suggests transforming the problem in a way of excluding undesired characteristics.

During the review of model-based parameter tuning approaches (Section~\ref{bg: parameter tuning}), we concluded that all broadly used system follows strictly the first approach. For instance, as the surrogate models, SMAC~\cite{hutter2011sequential} uses the random forest, BOHB~\cite{falkner2018bohb} and BRISE~\cite{brise2spl} — Bayesian probability models. While those surrogates could naturally fit to the described search space shape, none among proposed approaches is able to make the predictions effectively since the most of predicted configurations will violate the dependencies. As an instance, imagine after $i^{th}$ iteration, the surrogate models learn about two superior parameters: one indicates a well-performing heuristic type (the Genetic Algorithm), the other — an effective configuration for another algorithm type (an exponential cooling rate for the Simulated Annealing). In such a case, the reviewed systems sampling methods will tend to predict the configurations with those two parameter values, which turns to be invalid.

Here we follow the second approach namely, the transformation of the problem in order to sample the valid configurations only. The following section depicts a required preparation step, made in the search space, while the later is dedicated to the prediction process.


\section{Search Space}\label{concept:search space}
When the time comes to selecting not only the solver parameters, but also the solver itself, the united search space no longer could be presented as `flat' set of parameters since it tends to appearance vast amount of invalid parameter combinations.

Let us estimate the number of all possible configurations vs the amount of meaningful ones in rather simple example.
Suppose, we have $N_s$ solver types, each exposing the $N_{s,p}$ number of hyper-parameters with $N_{s,p,v}$ possible values. The aggregated quantity of configurations $N_c$ in the disjoint search spaces is calculated as the number of possible combinations using Formula~\ref{c: disjoint search space size}.

\begin{equation}
N_c = N_s \cdot \prod_{1}^{N_{s,p}} N_{s,p,v}
\label{c: disjoint search space size}
\end{equation}

However, if we decide to tune (or rather to control) the solver type itself, the resulting quantity of possible configurations is calculated using Formula~\ref{c: joint search space size}.

\begin{equation}
N_c = \prod_{1}^{N_{s}} \prod_{1}^{N_{s,p}} N_{s,p,v}
\label{c: joint search space size}
\end{equation}

For better intuition, lets try some numbers. By setting all $N_s = N_{s,p} = N_{s,p,v} = 3$ (the rather small example), the amount of configurations estimated separately for each solver equals to $N_c = 81$ (Formula~\ref{c: disjoint search space size}). However, if we join the solver parameter spaces, Formula~\ref{c: joint search space size} shows the significant growth in the search space size: $N_c = 19683$. Note, the number of different configurations remains the same thus, in the joint space it is only $\approx 0.4\%$. When setting $N_s = N_{s,p} = N_{s,p,v} = 4$, this number drops to $\approx 9 \cdot 10^{-8}\%$. It could decrease even further if the dependencies among $p$ exist. In such case, the predictive abilities of the parameter control approaches may straggle.

To overcome this, we utilize similar to the proposed in IRACE~\cite{lopez2016irace} framework idea: \emph{explicitly indicate the dependencies as a parent-child relationship among the search space entities $p$, firstly predict the parent parameter, afterwards — the children.} This give us an opportunity to threat the algorithm type as the regular categorical parameter and simplifies the structure of search space and makes the prediction process uniform.

This decision raises the following search space \emph{structural requirements}:
\begin{enumerate}[itemsep=8pt]
	\item[S.R.1] \textbf{Parent-child relationship} describe the dependencies between different parameter types. For instance, appearance of one requires another, but eliminates the other.

	\item[S.R.2] \textbf{Uniform parameter types} simplifies the structure and hides the domain-specific intent of each parameter thus, algorithm type and its configuration are treated in the same way.

	\item[S.R.3] \textbf{Value-specific dependencies} describe a concrete parent value(s), when the child (children) should be exposed. For instance, the entity \textit{algorithm type} has a number of possible values, each of them requires own set of hyper-parameters, which should remain hidden for the other solver types.
\end{enumerate}

\svgpath{{graphics/Concept/}}
\begin{figure}
	\centering
	\includesvg[width=1.0\textwidth]{feature tree}
	\caption{Search space representation.}
	\label{concept:pict:Search Space Representation}
\end{figure}

Figure~\ref{concept:pict:Search Space Representation} shows an example of such the search space with $s$ algorithm types, each having $p$ parameters with $v$ possible values. The entities with triangles $\bigtriangledown$, namely the parameter concrete values, form the joint-points to which the other parameters could be linked. 

\todoy{Should I mention that it is similar to SPL Feature Models? Only single sentence comes in my mind, but I am not sure if it is needed here: "This structure is similar to \emph{Feature Models}, used in Software Product Lines~\cite{bibid}."}

\section{Prediction Process}\label{concept:prediction}

After formalizing the structural requirements, let us switch to the prediction process and define the search space \emph{functional requirements}, which should be fulfilled to decouple the learning models from the complex search space shape.

The idea of this decoupling lays in resolving the value-specific dependencies among the parameters in a step-wise prediction process. To do so, we firstly predict the parent value, which in case of hyper-heuristic is the type of low-level heuristic (Level 0 on Figure~\ref{concept:pict:Level-wise prediction process}). Afterwards, the search space must expose the child parameters of this solver only, ignoring the others (Level 1 on Figure~\ref{concept:pict:Level-wise prediction process}). The dependencies among exposed parameters, are handled in the same way (Level 2 and further on Figure~\ref{concept:pict:Level-wise prediction process}).

While building the surrogates and making the prediction on each level, we need to ignore the data on the other levels. Making the prediction on parent level, we do not change it on levels below thus, we do not need to operate useless information. If the backward ignorance is painless, the forward omission puts the restriction on the surrogate models. Cutting off the deeper level parameter values, we may get the data points with the same parameters values, but different results. Thus, on this level(s) only those models should be used, which will not be confused. During the implementation description we will clarify, which models are the better choice in such cases and implement one of the promising.

Certainly, during the problem solving, the quality trends among parameter values change. For instance, at later stages the domination of one solver could be declined in comparison to other. Or, the previously good-performing parameter values are not that good anymore. It may be caused by the various reasons, but the old-trends should be definitely left out introducing some forgetting mechanism.

At this point, let us shortly clarify the resulting functional requirements, derived from the previous discussions:
\begin{enumerate}[itemsep=8pt]
	\item[F.R.1]
	
	\item[F.R.2]
	
	\item[F.R.3]
\end{enumerate}

\begin{figure}
	\centering
	\includesvg[width=1.0\textwidth]{feature tree pred}
	\caption{Level-wise prediction process.}
	\label{concept:pict:Level-wise prediction process}
\end{figure}

\paragraph{Importance explanation}
\paragraph{Requirements} generality, top-down approach of optimization
-- different views of same Configuration (level-dependent) - filtering, transformation
-- consider problem features? while selecting meta-heuristic \cite{kerschke2019automated} page 6
-- learning metrics (relative improvement), we postpone adding other metrics in future work.
IRACE\cite{lopez2016irace}

\section{Hyper-Heuristic}

\section{Low Level Heuristics}\label{concept: llh}
\paragraph{Importance explanation}
\paragraph{Requirements}


\section{Conclusion of concept}
to be done...
  \chapter{Implementation Details}
\paragraph{In this chapter} we dive into the implementation details of the selection hyper-heuristic with parameter control.
 
The best practice in software engineering is to minimize an effort for the implementation and reuse already existing and well-tested code.
With this idea in mind we had decided to reuse one of existing (and highlighted by us in \ref{bg: parameter tuning}) open-source hyper-parameter tuning systems as the code basis and those turn it into the core of hyper-heuristic. 
Do to so we analyze the existing systems and highlight important non-functional characteristics from the implementation perspective in section \ref{implementation:hh code basis section}. Since the selected code base system is not the ideal in terms of such features as Search Space entity abilities and the prediction process, we consider some adaptations in sections \ref{implementation: search space} and \ref{implementation: prediction logic} respectively.
We also reuse the set of Low Level Heuristics in section \ref{implementation:llh code basis selection}.


\section{Hyper-Heuristics Code Base Selection}\label{implementation:hh code basis section}
A.k.a. "brain". Need to find a better way to call this part of HH..
\subsection{Requirements}
\subsection{Parameter Tuning Frameworks}
\paragraph{SMAC}
\paragraph{BOHB}

% TODO: it is more for concept description
% Even if fix of Configuration sampling in such cases is relatively simple task (parent-child relationship utilized in IRACE), then building proper density distributions of `good' parameters reflecting conditions is hard.


\paragraph{IRACE}
\paragraph{BRISEv2}
\todoy{Maybe, smth else..}

% TODO: implementation \item[Support for Online Problem Solving.] This is a bit complex characteristic of system that we are willing to distinguish. As it turns out, most parameter tuning systems require full evaluation of Target System for Configuration comparison. However, in case of Hyper-Heuristic, the Configuration evaluation is a trial to solve the problem in hand using particular $LLH$ (tuned parameter) Here we compare final result quality, reported by each $LLH$
\subsection{Conclusion}
BRISEv2 is the best system for code basis, however it has to be changed as we describe in following sections.


\section{Search Space}\label{implementation: search space}

\subsection{Base Version Description}
What is the problem with the current Search Space?
\paragraph{The Scope Refinement Work} Throw away and write a new one :D

\subsection{Implementation}
\paragraph{Description}
\paragraph{Motivation of structure}
\paragraph{Class diagram} - i think, I will put it into the appendix


\section{Prediction Logic}\label{implementation: prediction logic}
\subsection{Base Version Description}
\paragraph{The Scope Refinement Work} prediction should be done in feature-tree structured search space. Most models could handle only flat search space and we would like to enable reuse of those existing models. Though we decouple the structure of Search Space in entity \textbf{Predictor}, while actual prediction process is done in underlying models, that Predictor uses.

\subsection{Predictor}
to decouple prediction from structure of search space.

\subsection{Prediction Models}
\subsubsection{Tree parzen estimator}
\subsubsection{Multi Armed Bandit}
\subsubsection{Sklearn linear regression wrapper}


\section{Data Preprocessing}
\subsection{Heterogeneous Data} description and motivation of data preprocessing notions

\subsection{Base Version Description} and Scope of work analysis
\subsection{Wrapper for Scikit-learn Preprocessors}


\section{Low Level Heuristics}\label{Impl: LLH}

\subsection{Requirements}

\subsection{Code Base Selection}\label{implementation:llh code basis selection}
Available Meta-heuristics with description of their current state
With the aim of effort reuse, the code base should be selected for implementation of the designed hyper-heuristic approach.
% https://docs.google.com/spreadsheets/d/19xjL_ire0R5VLP9seCE5_4sWorSMZP4xvgx7d1q4a9s/edit#gid=0
\paragraph{SOLID}
\paragraph{MLRose}
\paragraph{OR-tools}
\paragraph{pyTSP}
\paragraph{LocalSolver}
\paragraph{jMetalPy}

\subsection{Scope of work analysis}
\paragraph{opened PR}

\section{Conclusion}

  \chapter{Evaluation}\label{eval}
The concepts, proposed in \cref{Concept description}, implemented in \cref{impl} of search space representation, prediction process and based on this generalized parameter control approach, selection hyper-heuristic and the hyper-heuristic with parameter control should be broadly evaluated. The experiments may be performed in number of investigation directions, starting from the developed RL performance evaluation with respect to system configuration and ending with the scalability to different problem sizes.

The structure of this Chapter is as follows. We start with the optimization problem presentation in \cref{eval: op}, a short environment description in \cref{eval: environment} and parameter tuning of low-level heuristics in \cref{eval: mh tuning}, which will be used later through our tests. The evaluation process of the proposed concept could be divided into two main parts. 

The first part is dedicated to the developed concept analysis in comparison to the baseline and is presented in \cref{eval:1}. We start the concept evaluation with experiments planning in \cref{eval:1:plan} and proceed firstly reviewing the baseline in \cref{eval:1:baseline}, afterwards the generic parameter control is presented in \cref{eval:1:PC}, followed by selection hyper-heuristic with static hyper-parameters in low-level heuristics review in \cref{eval:1:hh-sp} and finally, the selection hyper-heuristic with parameter control in low-level heuristics review is presented in \cref{eval:1:hh-pc}.

In the second part of evaluation we investigate an influence of hyper-heuristic with parameter control settings on its performance in \cref{eval:2}. For doing so once again we firstly perform the experiment planning in \cref{eval:2:plan}. Afterwards, in \cref{eval:2:learning granularity} we investigate the influence of a learning granularity on a HH-PC performance, in \cref{eval:2:learning models} we perform changes of a learning models configurations and in \cref{eval:2:llh changes} we check the influence of possible changes in LLH behavior.

Finally, \cref{eval: conclution} concludes our discussion of the obtained results.


\section{Optimization Problem}\label{eval: op}
Through this thesis we are tackling a vehicle routing problem — the traveling salesman OP, which explanation could be found in \cref{BG: subsection OPs}. Nevertheless, as a reminder we repeat its short definition here. We also include the other details, related to the benchmarks.

``Given a set of cities and the distances among them, find the shortest path, which visits all cities''. It is a combinatorial OP with a number $n = N!$ of possible solutions. For the benchmarks we use several instances of symmetric TSP (distances $x_i \rightarrow x_j$ and $x_j \rightarrow x_i$ are equal) from a publicly available and broadly used benchmark set TSPLIB95~\footnote{TSPLIB95 website:~\url{http://comopt.ifi.uni-heidelberg.de/software/TSPLIB95/}}. The advantage of choosing this benchmark set lays in a broad compatibility of solvers and frameworks with the proposed standardized problem instance description (including used jMetal and jMetalPy). The TSP in this case is defined as a set of city coordinates therefore. Thus, before starting to solve a problem, the distance matrix should be built, calculating Euclidean distances between cities. For more detailed explanation of TSPLIB95 problem instance files refer to~\cite{reinelt1995tsplib95}.

For our benchmarks we select four problem instances from a simpler case harder they are: \emph{kroA100}, \emph{pr439}, \emph{rat783} and \emph{pla7397} of sizes 100, 439, 783 and 7397 cities respectively. The optimal tours for each of these problem instances were previously obtained by exact solvers and reported in aforementioned library. While presenting our evaluation results, we refer to them therefore, we present the optimal solution results in \cref{eval:table:tsp optimal tour length}.

\begin{table}[h!]
	\centering
	\begin{tabular}{c||c}
		\textbf{TSP instance} & \textbf{Optimal tour length} \\
		\hline
		\hline
		kroA100 & 21282 \\
		pr439 & 107217 \\
		rat783 & 8806 \\
		pla7397 & 23260728 \\
	\end{tabular}
	\caption{TSP instances optimal tour length.}
	\label{eval:table:tsp optimal tour length}
\end{table}

Please note that the goal of this thesis and the evaluation in particular is not to beat the exact solvers in any case, but to investigate the applicability of proposed generic parameter control concept.

\section{Environment Setup}\label{eval: environment}
To run our experiments we use an enhanced by our approach BRISEv2 and deploy it in Docker containers on a single host machine with following characteristics:
\begin{itemize}
	\item \textbf{Hardware:} Fujitsu ESPRIMO P958 computer with 64GB 2667MHz RAM (16GB * 4 pcs), Intel Core i7-8700 CPU @ 3.2 GHz (6 cores * 2 threads) and Samsung 1TB SSD.
	
	\item \textbf{Software:} GNU/Linux Fedora 29 host OS and installed docker version 1.13.1.
\end{itemize}

We deploy 6 homogeneous BRISEv2 workers with LLHs on the same host machine to carry the problem solving process.
We run each experiment 9 times for 15 minutes to obtain the statistical data.


\section{Meta-heuristics Tuning}\label{eval: mh tuning}
As we conclude in \cref{bg: parameter setting conclution}, the goal of parameter control is to reach at least the quality of parameter tuning approaches. Therefore, before running the major set of evaluation experiments, we have to perform a parameter tuning for the underlying LLHs.

\subsection{Parameter Tuning System Configuration.} 
As a tuning system, we used our the implemented concept but in the tuner mode. As we described in \cref{concept: conclution}, to enable the parameter tuning mode, we built a search space based on the singe LLH with its parameters. In our particular case it were three search spaces for each underlying meta-heuristic respectively. We also disabled the solution transfer between each configuration, forcing LLH to use the OP each time from scratch.

For each LLH we run the tuning for 8 hours on 10 deployed worker nodes and three minutes for task evaluation. The underlying prediction mechanism was configured to use TPE with 100\% window size. We also disabled the repetition strategy (\emph{repeater} entity), leaving each configuration evaluated only once (with one task). We do so since our preliminary experiments have shown that the variance among evaluations is negligible. As one may expect, since the repetition strategy was disabled, outliers detection was turned off as well.

\subsection{Target Optimization Problem and Search Space of Parameters.} 
The role of target optimization problem was played by one of evaluated TSP instances: \emph{rat783}. We selected this instance because, it is a middle size problem, comparing all the used through evaluation OPs.

\paragraph{jMetalPy evolution strategy.} This meta-heuristic is implemented in a framework as na\"ive evolution strategy however, we found an important recombination mechanism missing therefore, the heuristic is performing mostly by means of the mutation operations. As a configuration, this ES implementation requires providing several hyper-parameters. Integer $\mu$ (\emph{mu}), which denotes the number of parents in the population, while integer $\lambda$ (\emph{lambda}) defines the number of offspring. We tune both parameters in ranges $[1..1000]$. Boolean \emph{elitist} defines the selection strategy, which true value enables elitist selection $(\mu+\lambda)$, while false disables the elitist selection $(\mu,\lambda)$ (more detains in \cref{BG: MH Examples}). Also, the framework proposes two possible \emph{mutation types} for combinatorial OPs: permutation swap and scramble mutation, which we use for tuning. The respective mutation probability is tuned in range $[0..1]$.

\paragraph{jMetalPy simulated annealing.} In this meta-heuristic authors defined the solution neighborhood by means of the same mutation operators, mentioned above. Thus, we use them and the same mutation probability range for tuning the SA. Unfortunately, the authors did not provide other but exponential cooling schedule and did not expose parameters temperature or alpha. This is the reason of such tiny parameter space for this MH.

\paragraph{jMetal evolution strategy.} The set of exposed hyper-parameters is almost the same, as we described for the Python-based MH implementation. The only difference that the mutation is represented only by one type, therefore we exclude it from the parameter space but leaving the mutation probability. All the other parameter ranges are the same as for the defined above ES.


\subsection{Parameter tuning results.} 
The process of parameter tuning is depicted in \cref{eval:pict:mh tuning}. During the session each MH was probed with at least $1.5k$ configurations. 


\svgpath{{graphics/Eval/tuning}}
\begin{figure}[h!]
	\centering
	\includesvg[width=\textwidth]{tuning progress}
	\caption{The low level heuristics parameter tuning process.}
	\label{eval:pict:mh tuning}
\end{figure}

In the figures below we propose a visual analysis of the parameter tuning results. For each meta-heuristic we separately present the numerical and categorical parameters.

The numeric hyper-parameters are showed as scattered points of parameter value (\emph{x-axis}) and the respective objective function result (\emph{y-axis}), obtained for configuration with this parameter value. Although such an isolated approach to analyze data in some cases may be error-prone, still it enough to get a birds-eye view on the existing dependencies. To represent trends among numeric parameter values we draw the regression line ($4^{th}$ degree) in green. At the top and to the right of the graph presented also the axis value densities. Thus, the density on a right side shows which objective values and how often were obtained, changing the underlying parameter, while the density on the top shows which parameter values were selected more often.

As for the categorical parameters, we plot their values as violin plots. It is a combination of box plot with the addition of a kernel density plot on each side. Since in our case, all categorical parameters of underlying algorithms have only two values, each violin plot shows which results of an objective function and how often were obtained. Using colors we depict different value of underlying parameter, while the shape of violin shows an expected result value and its probability. Inside the figure we also draw three dashed lines. A middle line with long dashes is a median, while lower and upper lines with short dashes show first and third quartiles respectively.


\paragraph{jMetalPy evolution strategy parameters.}
\begin{figure}[h!]
	\centering
	\vspace{-20pt}
	\includesvg[width=\textwidth]{jMetalPy evolution strategy numeric parameters}
	\caption{jMetalPy evolution strategy numeric parameters values.}
	\label{eval:pict:jmetalpy es numeric}
	\vspace{-15pt}
\end{figure}

\begin{figure}[h!]
	\centering
	\vspace{-20pt}
	\includesvg[width=\textwidth]{jMetalPy evolution strategy categorical parameters}
	\caption{jMetalPy evolution strategy categorical parameters values.}
	\label{eval:pict:jmetalpy es categoric}
	\vspace{-20pt}
\end{figure}

Looking on the \cref{eval:pict:jmetalpy es numeric}, one will see an explicit dependency between the number of parents (\cref{eval:pict:jmetalpy es numeric} parameter \emph{mu}) and the objective function: less amount of parents are tended to produce the better results. However, the dependency is such clearly observable for the number of offspring (\cref{eval:pict:jmetalpy es numeric} parameter \emph{lambda}). We may see that a high number of offspring does not tend to provide good results, but the number of performed estimations for low \emph{lambda} is not enough to be strongly ensured that this value is better. Yet, even with small amount of observations we may make a guess that low \emph{lambda} is a good parameter choice. With respect to the mutation probability, it may be observed that the higher mutation rates tend to produce a better results. 

As for the categorical parameters, one may see a strong bias towards bad results when using non-elitist algorithm version (\cref{eval:pict:jmetalpy es categoric} parameter \emph{elitist}). When concerning the mutation type, the dominance is not an obvious, but permutation version of mutation is slightly outperforms scramble type (\cref{eval:pict:jmetalpy es categoric} parameter \emph{mutation}).


\paragraph{jMetalPy simulated annealing parameters.}
\begin{figure}[h]
	\centering
	\begin{subfigure}{0.35\textwidth}
		\vspace{-10pt}
		\includesvg[width=\linewidth]{jMetalPy simulated annealing numeric parameters}
		\caption{Mutation probability.}
		\label{eval:pict:jmetalpy sa numeric}
	\end{subfigure}
	\hfil 
	%\vspace{-5pt}
	\begin{subfigure}{0.4\textwidth}
		\includesvg[width=\textwidth]{jMetalPy simulated annealing categorical parameters}
		\vspace{-5pt}
		\caption{Mutation type.}
		\label{eval:pict:jmetalpy sa categoric}
	\end{subfigure}
	\caption{jMetalPy simulated annealing parameters.}
\end{figure}


This heuristic were tuned by means of only two parameters: categorical mutation type, which results are presented in \cref{eval:pict:jmetalpy sa categoric} and numerical mutation probability with graphs in \cref{eval:pict:jmetalpy sa numeric}. One may see a strong dominance of permutation mutation type, while scramble produce an average but stable results. The mutation probability trends are also clear: higher parameter values produce better results. Indeed, the dependency on mutation probability is obvious, since the underlying algorithm is performing the search space traversal by means of solution mutation. The two lines of results that could be viewed on the \cref{eval:pict:jmetalpy sa numeric} are correlated with the mutation type: lower corresponds to usage of permutation, while upper to scramble mutation.


\paragraph{jMetal evolution strategy parameters.}
\begin{figure}[h]
	\centering
	\vspace{-10pt}
	\includesvg[width=\textwidth]{jMetal evolution strategy numeric parameters}
	\caption{jMetal evolution strategy numeric parameters values.}
	\vspace{-15pt}
	\label{eval:pict:jmetal es numeric}
\end{figure}

\begin{wrapfigure}{R}{0.5\textwidth}
	\centering
	\vspace{-20pt}
	\includesvg[width=\linewidth]{jMetal evolution strategy categorical parameters}
	\label{eval:pict:jmetal es categoric}
	\caption{jMetal ES elitist parameter.}
	\vspace{-30pt}
\end{wrapfigure}

The final heuristic under investigation is the Java-based implementation of viewed above ES. Even if at the first glance the regression lines are not looking the same, the overall trends are similar: lower values of \emph{mu} parameter result in better objective, while the mutation probability should be kept high. On contrary to Python-based ES, here the middle-range values of parameter \emph{lambda} produce the best results. It may be explained by the fact of performance straggling in Python-based version: with large offspring number, the computational effort, required to accomplish the iteration increases, while Java-based version could handle it. A dominance of elitist version of algorithm is non-obvious, but this could be seen from a distribution first quartile.

\paragraph{}
We collected the best performing configurations of each meta-heuristic and presented them in \cref{eval: params jmetalpy es}. We also highlight here the default parameter values, which were selected with motivation of being in the middle of the values ranges.

\begin{table}%[h!]
	\centering
	\begin{tabular}{r||c|c|c}
		\textbf{Hyper-parameter} & \textbf{Default value} & \textbf{Tuned value} & \textbf{Estimated range} \\
		\hline
		\hline
		\rowcolor{gray!10}
		\multicolumn{4}{c}{jMetalPy evolution strategy} \\
		\hline
		$\mu$ & 500 & 5 & $[1..1000]$ \\
		$\lambda$ & 500 & 22 & $[1..1000]$ \\
		\emph{elitist} & False & True & {True, False} \\
		\emph{mutation type} & Permutation & Permutation & {Permutation, Scramble} \\
		\emph{mutation probability} & 0.5 & 0.99 & $[0..1]$\\
		\hline
		\rowcolor{gray!10}
		\multicolumn{4}{c}{jMetalPy simulated annealing} \\
		\hline
		\emph{mutation type} & Permutation & Permutation & {Permutation, Scramble} \\
		\emph{mutation probability} & 0.5 & 0.89  & $[0..1]$\\
		\hline
		\rowcolor{gray!10}
		\multicolumn{4}{c}{jMetal evolution strategy} \\
		\hline
		$\mu$ & 500 & 5 & $[1..1000]$ \\
		$\lambda$ & 500 & 605 & $[1..1000]$ \\
		\emph{elitist} & False & True  & {True, False}\\
		\emph{mutation probability} & 0.5 & 0.99 & $[0..1]$ \\
	\end{tabular}
	
	\caption{Static hyper-parameters of low-level meta-heuristics.}
	\label{eval: params jmetalpy es}
\end{table}


\section{Concept Evaluation}\label{eval:1}

\subsection{Evaluation Plan}\label{eval:1:plan}
To evaluate the performance of developed approach we firstly need to define the base line. In most cases it is the single meta-heuristics, which are solving the OP using static hyper-parameters. However, to evaluate the parameter control feature we must make a closer look on the performance of each separate heuristic with static and dynamic hyper-parameters. For selection hyper-heuristic analysis we compare the performances of all underlying MHs running separately and together within a hyper-heuristic. Note, in this case the hyper-parameters are statically defined. And last, but not least, to evaluate a selection hyper-heuristic with enabled parameter control we compare it to separately running underlying meta-heuristics with parameter control and to selection hyper-heuristic.

In order to organize the evaluation plan, we distinguish two stages.
At the first stage the LLH selection occurs, while at the second one we chose hyper-parameters for the selected LLH. At each stage we may use different prediction approaches, which description could be found in \cref{impl: prediction models}. To select the LLH, apart from random and static selection we also use FRAMAB (see \cref{impl: FRAMAB}) and Bayesian ridge regression model implementation from Scikit-learn framework (see \cref{impl: sklearn wrapper}). Note, for the Bayesian ridge regression model we use a default parameters, which could be found in the framework documentation\footnote{\href{https://scikit-learn.org/stable/modules/generated/sklearn.linear_model.BayesianRidge.html}{scikit-learn.org}}. To select the hyper-parameters for LLHs, apart from static default and tuned variants we also use random selection, available in BRISEv2 TPE and the mentioned above Bayesian ridge. The set of used techniques is presented in the \cref{eval: concept settings table}.
\begin{table}[h!]
	\centering
	\begin{tabular}{l||l}
		\textbf{LLH selection} & \textbf{LLH parameters selection} \\
		\hline
		\hline
		\textbf{1.} Random & \textbf{1.} Default \\
		\textbf{2.} Multi-armed bandit & \textbf{2.} Tuned beforehand \\
		\textbf{3.} Bayesian ridge regression & \textbf{3.} Random \\
		\textbf{4.1.} Static jMetalPy.ES & \textbf{4.} Tree Parzen Estimator (TPE) \\
		\textbf{4.2.} Static jMetalPy.SA & \textbf{5.} Bayesian ridge regression (BRR) \\
		\textbf{4.3.} Static jMetal.ES & 
	\end{tabular}
	
	\caption{Prediction techniques used for the concept evaluation.}
	\label{eval: concept settings table}
\end{table}


Using this table, we now could pick a prediction technique to form a desired system configuration. For instance, mentioned above baseline could be encoded into configurations starting from \emph{4.1.1} for the evolution strategy from jMetalPy framework, running with default hyper-parameters and ending with \emph{4.3.2} for evolution strategy from jMetal framework, running with tuned beforehand parameters.

Our benchmark plan for the concept evaluation looks as a set of following experiment groups:
\begin{itemize}
	\item \textbf{Meta-heuristics (MH).} The baseline. We evaluate each used meta-heuristic separately with default and tuned hyper-parameters: \emph{4.1.1} and \emph{4.1.2} for jMetalPy evolution strategy;  \emph{4.2.1} and \emph{4.2.2} for jMetalPy simulated annealing; \emph{4.3.1} and \emph{4.3.2} for jMetal evolution strategy respectively.

	\item \textbf{Meta-heuristics with parameter control (MH-PC).} The set of experiments dedicated to verify an impact of the generic parameter control on meta-heuristics performance. A selected set of experiments looks as follows: \emph{4.1.3, 4.2.3, 4.3.3} to investigate the influence of random parameter allocation; \emph{4.1.4, 4.2.4, 4.3.4} to check TPE-based parameter control and \emph{4.1.5, 4.2.5, 4.3.5} to probe Bayesian-ridge-based parameter control.

	\item \textbf{Selection hyper-heuristic with static parameters (HH-SP).} These benchmarks are dedicated to an investigation of the implemented on-line selection HH performance. It implies the LLHs usage with static parameters therefore, we evaluate HH-SP performance with default and tuned beforehand LLH parameters. Experiment codes are following: \emph{2.1, 2.2} for FRAMAB-based HH-SP and \emph{3.1, 3.2} for Bayesian-ridge-based HH-SP.
	
	\item \textbf{Selection hyper-Heuristic with parameter control in LLH (HH-PC).} This is a final set of benchmarks for concept evaluation. By this we evaluate an influence of simultaneous on-line LLH selection and parameter control on system performance. The respective experiment set is following: \emph{1.3, 2.4, 2.5, 3.4, 3.5.}
\end{itemize}

The aggregated concept benchmark plan is presented in \cref{eval: concept benchmark plan table}. The required running time of this experiment set is approximately 9 days and 18 hours on a single machine.
\begin{table}[h!]
	\centering
	\begin{tabular}{c||p{3cm}}
		\textbf{Experiment group} & \textbf{Related codes} \\
		\hline
		\hline
			
		\multirow{2}{*}{MH} & 4.1.1., 4.2.1, 4.3.1 \newline 4.1.2, 4.2.2, 4.3.2 \\
		
		\rowcolor{gray!10}
		\multirow{3}{*}{MH-PC} & 4.1.3, 4.2.3, 4.3.3 \newline 4.1.4, 4.2.4, 4.3.4 \newline 4.1.5, 4.2.5, 4.3.5 \\
		
		\multirow{3}{*}{HH-SP} & 1.1, 1.2 \newline 2.1, 2.2 \newline 3.1, 3.2 \\

		\rowcolor{gray!10}
		\multirow{3}{*}{HH-PC} &  1.3 \newline 2.4, 2.5 \newline 3.4, 3.5 \\
	\end{tabular}
	
	\caption{Concept benchmark plan.}
	\label{eval: concept benchmark plan table}
\end{table}


\subsection{Baseline Evaluation}\label{eval:1:baseline}
As we discussed previously, our results comparison should be done against the defined baseline. Therefore, this section is dedicated to review of the meta-heuristics performance out-of-the-box on different problem sizes and parameter settings. For visibility reasons we plot the intermediate and the final performance evidences for each problem instance separately, since they naturally imply different result ranges.

Since we are tackling a set of TSP instances, which were previously solved by other exact solvers, we also present an optimal solution, available for each instance as a green doted line.

%\newpage
\paragraph{kroA100 and pr439 TSP instances.}

Both TSP for 100 and 439 cities are a relatively small problem instances. Therefore, all underlying MHs reach a local optimum after few first external iterations. The difference between the external and internal iteration is following: the first one happens, when the main node selects configuration and sends it to the worked node, which carries out the LLH (MH) execution (see detailed description in \cref{impl}). On a contrary, the internal iteration occurs inside LLH itself, since it is an any-time algorithm. Thus, while the MHs reach a local optimum, there is no reason to spend much time for such cases review. We put a visual representation of benchmarks for kroA100 and pr439 into the thesis appendix (\cref{app:eval:bl plots}).

The only worth to mention observation is a worse SA results with tuned parameters, in contrast to default values on kroA100 TSP instance. It is explained by the fact that for algorithm tuning we used a different problem instance (rat783). It only confirms a motivation of the parameter control approaches: tuning is not problem-instance-universal technique.

\paragraph{rat783 TSP instance.}
\svgpath{{graphics/Eval/baseline}}
\begin{figure}[b]
	\centering
	\vspace{-20pt}
	\includesvg[width=\textwidth]{rat783 baseline progress}
	\caption{Intermediate results of meta-heuristics with static parameters on rat783.}
	\vspace{-5pt}
	\label{eval:pict:bl:rat783 intermediate}
\end{figure}

\setlength{\columnsep}{5pt}%
\setlength{\intextsep}{5pt}%
\begin{wrapfigure}{R}{0.5\textwidth}%\setcapindent{1em}
	\centering
	\includesvg[width=\linewidth]{rat783 baseline final boxplot}
	\label{eval:pict:bl:rat783 final}
	\caption{Final results of meta-heuristics with static parameters on rat783.}
	\vspace{-10pt}
\end{wrapfigure}
This is an average size problem among reviewed in the thesis. In contrast to previous instances, a behavior of solvers in this case changes slightly. For instance, ES from jMetal framework (j.ES) with default parameters is performing extremely slowly on a problem, however, while using an optimized hyper-parameters it quickly reaches a local optima (after ~50 iterations) with the best produced results among other heuristics (\cref{eval:pict:bl:rat783 intermediate}). Analyzing performance evidences of Python-based solvers, we may conclude that they almost reached a local optima in a given 15 minutes, therefore, their final results are slightly worse than the produced by j.ES (\cref{eval:pict:bl:rat783 final}).

\textbf{Please note}, the perturbation of trends in the end of tuned py.ES run is caused by the difference in number of iterations among all runs (left figure in \cref{eval:pict:bl:rat783 intermediate}, tuned parameters). The software~\footnote{Python Seaborn data visualization framework web page:~\href{https://seaborn.pydata.org/}{seaborn.pydata.org}}, used in this thesis for results presentation estimates an average and the result deviation at each iteration. Thus, if one (or several) experiment execution(s) managed to perform more iterations than the majority of others, the average among and their deviation will be changed respectively. The number of such external iterations varies, since as a termination criterion we used the wall-clock time. Unfortunately, this perturbation appear in most of the \emph{progress charts}, therefore the comparison of final results is done by means of presented separately box-plots.


The bold lines in \cref{eval:pict:bl:rat783 intermediate} is a statistical mean of all 9 experiment runs with default parameter values (blue line) and tuned parameter values (orange line). A shadow around these lines is a confidence interval. One may observe how differently the parameter setting affects MHs: in evolution strategies the changes in performance is dramatic, while simulated annealing is almost not affected (\cref{eval:pict:bl:rat783 final}). We also observe a decreased stability of tuned jMetalPy ES meta-heuristic (py.ES). It is reflected in a large confidence interval not only of the intermediate results, but also in a statistic of final solution quality.


\paragraph{pla7397 TSP instance.} 

\begin{figure}[b]
	\centering
	\vspace{-20pt}
	\includesvg[width=\textwidth]{pla7397 baseline progress}
	\caption{Intermediate results of meta-heuristics with static parameters on pla7397.}
	\vspace{-5pt}
	\label{eval:pict:bl:pla7397 intermediate}
\end{figure}

\begin{wrapfigure}{R}{0.5\textwidth}%\setcapindent{1em}
	\centering
	\includesvg[width=\linewidth]{pla7397 baseline final boxplot}
	\label{eval:pict:bl:pla7397 final}
	\caption{Final results of meta-heuristics with static parameters on pla7397.}
	\vspace{-10pt}
\end{wrapfigure}

The largest investigated here TSP instance for ~$7.4k$ cities is, however, referred as a middle-size OP in used TSPLIB95. In this case the performance evidences changed the most, therefore, we discuss each MH behavior separately.

Python-based version of ES provide the worst results with both default and tuned parameters. Note the number of performed iterations by this MH with default parameters in a given 15 minutes is less 50. It is affected by a several reasons. Firstly, the amount of time required to perform an internal iteration increased dramatically. Thus, with specified 15 seconds for one task run, it actually takes much more time (up to 1 minute) to accomplish the task. We have a several guesses, what caused such a behavior. Firstly, it is a general Python performance issues, in most of the times caused by a global interpreter lock (GIL) usage. The explanation of GIL is out of this thesis scope, but roughly speaking, a multi-threading in Python causes struggle in comparison with single-threaded execution. Secondly, it may be caused by the algorithm code basis implementation, performed in jMetalPy framework. To implement a generic termination criteria (and some other features) the authors utilized a push-observer design pattern~\cite{benitez2019jmetalpy} according to which the underlying algorithm (ES in our case) triggers its observers after finishing each iteration. Therefore, stopping criteria may be evaluated only after finishing this internal iteration, which in case of running py.ES with TSP instance for $7.4k$ cities, may take a while depending on the algorithm configuration. For instance, with parameters \texttt{\{mu=5, lambda=10\}} the algorithm will terminate in time, while setting \texttt{\{mu=500, lambda=500\}} the algorithm will perform a higher number of computations on arrays of $7.4k$ integers long and as a result, may struggle. We observed the ES algorithm termination after a very first internal iteration. This also causes a poor solution quality improvements. Naturally, there is also an overhead in the results sending through the network, but as we observe on the Java-based ES performance with default parameter values, this overhead caused the decrease only in ~20 external iterations. We also eliminate the possible issue in a required time for problem loading (building the TSP distance matrix), since with caching implemented in jMetalPy wrapper (see \cref{impl: LLH scope}), the worker node does it only once and stores its cached version. In any case, this issue requires deeper investigation that we postpone to the future work. Running jMetalPy ES with tuned parameters fixes the issue with task reporting delays and therefore, results in higher number of external iterations, but the solution improvements are still weak (see left picture on the \cref{eval:pict:bl:pla7397 intermediate}).

As in the previous case, jMetalPy simulated annealing produce good quality improvements at each external iteration, least depending on the hyper-parameter values. Even with a default configuration, py.SA outperforms the final results of py.ES after the first 50 external iterations. Setting the tuned parameter values, the performance of algorithm increases, but not dramatically. A resulting progress curve, presented in the middle of \cref{eval:pict:bl:pla7397 intermediate} shows that py.SA requires more time to converge that was provided and is still far from its potential local optima.

jMetal evolution strategy is a perfect candidate to show, how important is a parameter setting. With default configuration j.ES is struggling in making improving steps and can not compete with other algorithms. Our guess here is the same as for the Python-based version: the number of internal iterations is extremely low for making a good search space traversal. However, a tuned version of j.ES outperforms all other solvers (see \cref{eval:pict:bl:pla7397 intermediate} and \cref{eval:pict:bl:pla7397 final} respectively).


\paragraph{Discussion.} The observed results of meta-heuristics execution confirm the algorithm parameter setting problem importance, discussed in \cref{bg: section Parameters Setting}. An effect of proper parameter selection is different among algorithms. In our case, the performance of two out of three solvers are highly dependent on the hyper-parameter settings. Thus, an application of the proposed in \cref{Concept description} generic parameter control approach to these algorithms is rather intriguing and may partially reveal the overall methodology benefits.

From the other side, we observe of only one algorithm domination among the others with static parameters. See how all MHs were solving each TSP instance with default parameters: in each case SA outperforms two other ES. The usage of all three MHs in a selection hyper-heuristic with static hyper-parameters will reveal the implemented approach applicability. In this case we expect to observe the results close to provided by pure SA. On a contrary, when we switch to tuned parameter usage, j.ES is preferred. We see it clearly when the MHs are applied to the biggest TSP instance with tuned hyper-parameters. Thus, we expect to observe such a behavior of selection hyper-heuristic.

%\newpage
\subsection{Generic Parameter Control}\label{eval:1:PC}
As we discussed in \cref{bg: parameter control}, the goal of parameter tuning lays in adaptive changing of underlying algorithm parameters with to optimize some performance measurement. In our case, we apply the proposed in \cref{concept: conclution} methodology to set the parameters of meta-heuristics in a runtime. Here is a brief reminder: at each RL step HLH is analyzing the past performance evidences of solver depending on its configuration to choose the parameters values, which hopefully lead to the higher solution quality improvements. Afterwards, we run the solver with sampled parameters for a predefined time (15 seconds) to get new evidences, attaching the previously best obtained solutions. Please note, according to our setup, we have 6 simultaneously running and reporting workers.

In this part of evaluation we compare the performance of algorithms with statically defined default and tuned hyper-parameters to dynamically changing parameter values by means of RL control.

\paragraph{kroA100 TSP instance.}
\svgpath{{graphics/Eval/control}}
\begin{figure}[t]
	\centering
	\vspace{-20pt}
	\includesvg[width=\textwidth]{kroA100 PC progress}
	\caption{Intermediate results of meta-heuristics with parameter control on kroA100.}
	\vspace{-5pt}
	\label{eval:pict:pc:kroA100 intermediate}
\end{figure}
\begin{figure}[b]
	\centering
	\vspace{-20pt}
	\includesvg[width=\textwidth]{kroA100 PC final boxplot}
	\caption{Final results of meta-heuristics with parameter control on kroA100.}
	\vspace{-5pt}
	\label{eval:pict:pc:kroA100 final}
\end{figure}

Comparing to the baseline, parameter control in a small problem instance was able to reach and even outperform the results of MHs on static parameters after first 50 iterations (\cref{eval:pict:pc:kroA100 intermediate}). We may observe that even random changes of the heuristic parameters in a runtime results in finding a better solutions comparing with statically defined case (\cref{eval:pict:pc:kroA100 final}). It is caused by the changes in a neighborhood definition (mutation type) and traversal process (mutation probability). In most cases, given enough time the learning-based parameter assignment outperforms random allocation.

Note the amount of configurations (iterations) performed by jMetal ES, which could be seen in \cref{eval:pict:pc:kroA100 intermediate}. According to our plan, a given time for MH run is 15 minutes, 15 seconds for running one configuration on 6 available workers. Thus, in the most optimistic case, the number of iterations should be $ \frac{15\cdot60\cdot6}{15} = 360$ but, we observe even more than 400. After an investigation, we came to conclusion that it is caused by an implementation flaw an insight of which is following. jMetal MHs provide only iteration-number-based termination criterion, which is not encapsulated how it is done in jMetalPy. For our needs we added also a time-based but did not remove the previously existing. For the iteration counter used a regular integer number, which we set up to its maximal value when using a time-based criterion. Given a specific `light' algorithm configuration (low $\mu$, $\lambda$ and mutation probability), with this OP MH is able to reach the maximal number of iteration in less than 15 seconds therefore, terminating early and triggering a new parameter control iteration. Certainly, it is our implementation flaw, which should be fixed in a future work.

%Let us have a closer look on the jMetal ES. Two observations could be made. Firstly, the stability of results in case of random-based parameter tuning weak in comparison to model-based, which is a reasonable behavior. However, in case of jMetalPy ES, the TPE-based stability is less than random-based. This leads us to 


\paragraph{pr439 and rat783 TSP instances.}
\begin{figure}[t]
	\centering
	\vspace{-20pt}
	\includesvg[width=\textwidth]{rat783 PC progress}
	\caption{Intermediate results of meta-heuristics with parameter control on rat783.}
	\vspace{-5pt}
	\label{eval:pict:pc:rat783 intermediate}
\end{figure}
\begin{figure}[b]
	\centering
	\vspace{-20pt}
	\includesvg[width=\textwidth]{rat783 PC final boxplot}
	\caption{Final results of meta-heuristics with parameter control on rat783.}
	\vspace{-5pt}
	\label{eval:pict:pc:rat783 final}
\end{figure}

In this two cases, the behavior of solvers were similar, therefore, we decided to join their discussion and present only the plots for a larger instance (rat783). Still, the graphical representation of intermediate and final performance on pr439 TSP instance is presented in \cref{app:eval:pc plots}.

When the parameter control is applied to a larger problem, it is starting to require more evidences for finding a good-performing settings of jMetalPy evolution strategy. Concretely, only the TPE-based parameter control was the closest in approaching the solution quality of tuned parameters. All techniques produced highly unstable intermediate and final results: please, draw your attention to the left side of \cref{eval:pict:pc:rat783 intermediate}, and filled with blue boxes in \cref{eval:pict:pc:rat783 final} respectively.

The results of parameter control application to less sensitive py.SA are following: randomized parameter sampling settled on the level of default parameters quality. BRR-based parameter control yielded a slightly better results, but not such stable, while TPE model approached the quality of tuned parameters (\cref{eval:pict:pc:rat783 final}).

On contrary, applying generic parameter control to j.ES MH leads to a comparable with the quality of tuned parameters (\cref{eval:pict:pc:rat783 final}). Note, as for previous problem instance, even a random-based parameter sampling outperforms default parameters.

\paragraph{pla7397 TSP instance.}
\begin{figure}[t]
	\centering
	\vspace{-20pt}
	\includesvg[width=\textwidth]{pla7397 PC progress}
	\caption{Intermediate results of meta-heuristics with parameter control on pla7397.}
	\vspace{-5pt}
	\label{eval:pict:pc:pla7397 intermediate}
\end{figure}

\begin{figure}[b]
	\centering
	\vspace{-20pt}
	\includesvg[width=\textwidth]{pla7397 PC final boxplot}
	\caption{Final results of meta-heuristics with parameter control on pla7397.}
	\vspace{-5pt}
	\label{eval:pict:pc:pla7397 final}
\end{figure}

For the final problem instance we omit the parameter control results of py.ES, since it did not manage to perform even slightest improvement in comparison to the default parameter values, presented in a baseline description. It is caused by a fact that in early stages our approach acts as a random search, since not enough evidences were obtained to build a prediction models. Thus, is case of py.ES, the MH was running with badly performing configurations and as we explained in \cref{eval:1:baseline}, did not manage to perform enough iterations to improve solution in a given 15 minutes.

As in previous cases, the least parameter-settings-sensitive py.SA shows an ability to perform almost equally with any parameter settings (\cref{eval:pict:pc:pla7397 intermediate}). Neither among available control techniques was able to outperform tuned beforehand algorithm configuration in the final results quality (\cref{eval:pict:pc:pla7397 final}). Among used models, only the BRR shabbily settled in the middle between default and tuned parameters quality.

As for j.ES, model-based approaches are outperforming the randomized parameter values allocation, not even talking about default parameters. Moreover, TPE-based control outperformed even the results of tuned beforehand hyper-parameters.


\paragraph{Discussion.} In general, the review of meta-heuristic performance on different problem instances showed that the proposed generic parameter control approach is applicable and able to yield not only the near-tuned parameters quality, but in sometimes even outperforming results: all MHs with kroA100 TSP instance jMetal evolutionary strategy with pr439 and pla7397.

Taking into account the results with random parameter allocation we are making two conclusions. Firstly, even a randomized parameters changes are able to improve a potentially bad static hyper-parameter setting (j.ES case). Secondly, the learning mechanisms should and must be improved further by means of different surrogate models usage and proper technique for surrogates optimization to more resize parameter search. Leaving the improvement steps to future work we conclude that the proposed generic parameter control concept is able to produce better results while solving an unforeseen problem on-line.


\subsection{Selection Hyper-Heuristic with Static LLH Parameters}\label{eval:1:hh-sp}
The second mode of developed approach and at the same time a main goal of this thesis is a process of dynamic heuristic selection. It was implemented in form of RL-based on-line selection hyper-heuristic, described in \cref{concept: conclution}. Here we use three available LLHs (mentioned above py.ES, py.SA and j.ES) with static parameter (default and tuned) into selection hyper-heuristic (HH-SP). Three approaches to select the LLH were investigated: randomized, FRAMAB- and BRR-based HLH.

We present the problem solving process in two forms. Firstly, we distinguish the selected at each iteration LLH and the results, which it gave. For doing so, we took only the first repetition (out of 9 available). Secondly, we present the final results of all runs in form of box-plots, comparing them to the underlying LLHs performance. The left group of box-plots presents the final solution quality, obtained with the default parameter values, while on the right site the results of tuned beforehand LLHs are outlined.

\paragraph{kroA100, pr439 and rat783 TSP instances.}
\svgpath{{graphics/Eval/selection}}
\begin{figure}[t]
	\centering
	\vspace{-20pt}
	\includesvg[width=\textwidth]{rat783 HH-SP progress}
	\caption{Intermediate performance of HH-SP on rat783 (single experiment).}
	\vspace{-5pt}
	\label{eval:pict:hh-sp:rat783 intermediate}
\end{figure}
\begin{figure}[b]
	\centering
	\vspace{-20pt}
	\includesvg[width=\textwidth]{rat783 HH-SP final boxplot}
	\caption{Final results of HH-SP on rat783 (statistic of 9 runs).}
	\vspace{-5pt}
	\label{eval:pict:hh-sp:rat783 final}
\end{figure}
Once again, we are grouping relatively small problem instances, on which the implemented HH-SP performs similarly. For the analysis we selected the largest instance among them: rat783. Nevertheless, the figures depicting kroA100 and pr439 TSP instances may be found in \cref{app:eval:hh-sp}.

% default parameters
Firstly, we would like to draw the reader's attention to HH-SP cases, in which the LLHs are used with the default parameter values (upper row in \cref{eval:pict:hh-sp:rat783 intermediate}). We do so, while according to the presented earlier baseline evaluation, there is only one algorithm with a strong performance dominance: py.SA. Thus, in \cref{eval:pict:hh-sp:rat783 intermediate} we may observe a high frequency of py.SA sampling by both learning-based selection strategies. One may distinguish a repetitive pattern in LLH allocation with FRAMAB (middle column). It is caused by a deterministic essence of the algorithm. When it reaches the critical point of changing the favor of utilizing py.SA and exploiting other heuristic, the FRAMAB's exploration mechanism fully guides a selection. Due to the usage of a similar time-based LLH termination, all workers are starting the next round in bunches. Thus, when a new round starts, FRAMAB operates on static information and allocates all next configurations with the same LLH, which turns to be the second best performing LLH: j.ES. Therefore, in such a setup we conclude FRAMAB behaves slightly inertly. One may argue this will cause a performance struggling, which is rather a logical conclusion, but it requires a further investigation, which we are forced to postpone for the future work due to a lack of time. In case of BRR usage (right column in \cref{eval:pict:hh-sp:rat783 intermediate}), the bias is strongly shifted towards j.ES, which in some cases may cause performance issues due to lack of exploration. According to presented in \cref{eval:pict:hh-sp:rat783 final} final results statistics, given at least one dominating LLH (default parameters values case), even a random LLH selection could utilize enough times to obtain a good solution quality, however, it may require more time to converge. The model-based LLH selectors produce a better results quality.

The next setup is a set of LLHs with tuned parameters (lower row in \cref{eval:pict:hh-sp:rat783 intermediate}). According to the baseline evaluation, all among available LLHs are able to tackle the problem and produce a similar solution quality, however, the light difference in a dominance present and is as following (descending): j.ES, py.SA, py.ES. As a consequence, FRAMAB HLH frequently utilizes the best j.ES (see middle of lower row in \cref{eval:pict:hh-sp:rat783 intermediate}). On contrary, BRR HLH, similarly to random-based, samples all LLH types almost evenly. We may conclude that BRR is not as sensitive to performance evidences and was `confused' since the process quickly converged into a local optima. The quality of final result presented in \cref{eval:pict:hh-sp:rat783 final} of all HP-SP versions with tuned LLHs are at least as good, as the solution quality provided by the best underlying LLH due to its fast convergence (see \cref{eval:pict:bl:rat783 intermediate}).

\paragraph{pla7397 TSP instance.}
\begin{figure}[t]
	\centering
	\vspace{-20pt}
	\includesvg[width=\textwidth]{pla7397 HH-SP progress}
	\caption{Intermediate performance of HH-SP on pla7397 (single experiment).}
	\vspace{-10pt}
	\label{eval:pict:hh-sp:pla7397 intermediate}
\end{figure}

\begin{figure}[b]
	\centering
	\vspace{-20pt}
	\includesvg[width=\textwidth]{pla7397 HH-SP final boxplot}
	\caption{Final results of HH-SP on pla7397 (statistic of 9 runs).}
	\vspace{-5pt}
	\label{eval:pict:hh-sp:pla7397 final}
\end{figure}
Our observations of different MH heuristics on the largest tackled in this thesis problem are as following. During the baseline evaluation py.ES with default parameters had the worst performance and not able to accomplish external iteration in time, while the tuned algorithm version was able to outperform only default j.ES. On contrary, j.ES with tuned parameters produced the best results, outperforming both middle-quality py.SAs. The best performing meta-heuristic with default parameters was py.SA. According to this information, we observe an expected behavior of HH-SP with default LLHs (upper row in \cref{eval:pict:hh-sp:pla7397 intermediate}): the most frequently sampled by learning-based HLH was py.SA. However, the number of py.ES usages is suspiciously high in BRR case. According to baseline evaluation, the optimization process was not able to advance in a search using the py.ES (which we observe in a random-based sampling case, upper left cell in \cref{eval:pict:hh-sp:pla7397 intermediate}). Till the present time we do not have a comprehensive explanation of this behavior. Referring to the final results, presented in \cref{eval:pict:hh-sp:pla7397 final} we observe a high diverse in quality when py.ES is allocated frequently (codes 1.1. and 3.1.), which is an expected behavior, taking into account the performance of py.ES with default parameters.

Talking about the tuned LLHs case, we observe almost equal performance of all LLH sampling approaches, comparable to the best available LLHs final results. The solution quality of random-based HLH is slightly worse, in comparison to the results of FRAMAB- and BRR-based HH-SP due to their frequent usage of j.ES (see right chart in \cref{eval:pict:hh-sp:pla7397 final}).

\paragraph{Discussion.} According to our observations of the developed selection hyper-heuristic performance we conclude that the proposed concept implementation operates as expected: HH-SP exposes similar performance to the best available underlying LLH. Two implemented selection HLH are performing slightly differently when reaching a local optimum. We claim the FRAMAB is a more perspective HLH, since it starts to balance between previously seeing good performing LLH exposing a good exploration abilities. On a contrary, BRR continues to utilize only the best performing heuristic. In case when the advantage of one LLH changes to another, BRR may need more time to learn this. Nevertheless, our guesses require more evaluation proves.

The observed issues call for not only a thorough investigation (pla7397 code 1.1, 3.1), but also a generic approach to handle a potential flaws in LLH implementation that may cause struggling of overall HH-SP execution. The implemented system should be evaluated by means of HLH configuration influence on the performance. Also, a further investigation of adding several new LLHs should be evaluated by means of required computation effort for finding a good LLHs vs pure performance of these LLHs.


\subsection{Selection Hyper-Heuristic with Parameter Control}\label{eval:1:hh-pc}
The final evaluation is dedicated to performance analysis of the suggested approach of merging the selection hyper-heuristic with the generic parameter control technique. A minimal goal is to obtain a performance of the best underlying LLH algorithm with tuned hyper-parameters. In this evaluation set we follow a similar to used for HH-SP method of intermediate results review distinguishing allocated LLH types at each iteration for one repetition and comparing the quality of final results over all repetitions with a baseline. As specified in the evaluation plan (\cref{eval:1:plan}), for the LLH selection we use three approaches: random, FRAMAB and BRR sampling, while for LLH parameter control we use the random, TPE and BRR sampling.

\paragraph{kroA100, pr439 and rat783 TSP instances.}
\svgpath{{graphics/Eval/hhpc}}
\begin{figure}[t]
	\centering
	\vspace{-20pt}
	\includesvg[width=\textwidth]{rat783 HH-PC progress}
	\caption{Intermediate performance of HH-PC on rat783 (single experiment).}
	\vspace{-5pt}
	\label{eval:pict:hh-pc:rat783 intermediate}
\end{figure}

\begin{figure}[b]
	\centering
	\vspace{-20pt}
	\includesvg[width=\textwidth]{rat783 HH-PC final boxplot}
	\caption{Final results of HH-PC compared with MH on rat783 (statistic of 9 runs).}
	\vspace{-5pt}
	\label{eval:pict:hh-pc:rat783 final}
\end{figure}

The decision to join all three TSP instances is motivated by a similar to mentioned in HH-SP discussion reasons: the system intermediate and final performance are rather similar among problem instances, therefore, here we review only a single (rat783) case. The results for all other problems may be found in~\cref{app:eval:hh-pc}.

During the solving process a similar to HH-SP patterns of algorithm allocation may be observed for both FRAMAB-based (codes 2.4, 2.5) and BRR (codes 3.4, 3.5) HLHs. However, in this case the intermediate results are slightly differing, since the parameter control has started to search for good LLHs settings (\cref{eval:pict:hh-pc:rat783 intermediate}). 

Let us firstly draw the reader's attention to HH-PC with FRAMAB for LLH and TPE for parameter sampling (code 2.4 in \cref{eval:pict:hh-pc:rat783 intermediate}). At the beginning of solving process, j.ES was performing extremely well, for this reason FRAMAB was sampling it with a higher frequency. When a solving process reached its local optima (nearly $50^{th}$ iteration), FRAMAB started exploration of the other heuristics. The appeared `noise' in the results of both j.ES and py.ES heuristics is caused by a boolean parameter \emph{elitist}, which defines the selection strategy and may result in the solution quality degradation (a more detailed description of ES parameters could be found in \cref{BG: MH Examples}). From an absence of noise in later stages of 2.4 and 3.4 its may be seeing that TPE has found \emph{elitist=False} parameter value to be perspective in both ESs. On a contrary, BRR-based parameter controller did not find these parameters and glancing on the quality of final results (\cref{eval:pict:hh-pc:rat783 final}) we conclude the BRR statistically finds worse performing parameters, comparing to TPE on rat783 problem instance. Also, we must not ignore the fact of early reaching a local optimum by the search process (see shapes of progress curves in \cref{eval:pict:hh-pc:rat783 intermediate}). In such case, the parameter search should be biased towards exploration. For that we need to introduce other progress metrics, such as stagnation detection (which is used in EA parameter control in~\cite{karafotias2014generic}) and perform multi-objective RL optimization, maximizing improvement and minimizing stagnation. We postpone this enhancement for a future work due to a lack of time.

In \cref{eval:pict:hh-pc:rat783 final} we clearly see the dominating j.ES MH with tuned parameters (code 4.3.2). The quality of final solution, produced by the proposed concept implementation is statistically slightly lower the best performing tuned j.ES. This may be explained by a lack of time for parameter control to find a good performing setting, since the optimization reached its local optima too quickly to find good parameters for the underlying j.ES.


\paragraph{pla7397 TSP instance.}
\begin{figure}[t]
	\centering
	\vspace{-20pt}
	\includesvg[width=\textwidth]{pla7397 HH-PC progress}
	\caption{Intermediate performance of HH-PC on pla7397 (single experiment).}
	\vspace{-5pt}
	\label{eval:pict:hh-pc:pla7397 intermediate}
\end{figure}

\begin{figure}[t]
	\centering
	\vspace{-20pt}
	\includesvg[width=\textwidth]{pla7397 HH-PC final boxplot}
	\caption{Final results of HH-PC compared with MH on pla7397 (statistic of 9 runs).}
	\vspace{-5pt}
	\label{eval:pict:hh-pc:pla7397 final}
\end{figure}

As we observed in previous experiments, py.ES with non-tuned parameters causes struggling in hyper-heuristic performance (see \cref{eval:pict:bl:pla7397 intermediate} and the discussion below \cref{eval:pict:pc:pla7397 intermediate}). That is why during these benchmarks py.ES made a crucial change in the overall number of iterations where it was used many times (codes 1.3, 3.4, 3.5 in \cref{eval:pict:hh-pc:pla7397 intermediate}). While the case of fully randomized HH-PC (code 1.3) is clear, a BRR LLH selection did use this LLH frequently because it may have produced a good solutions quality as a result of parameter control behavior. On a contrary, FRAMAB LLH selection in combination with TPE parameter tuning (code 2.4) managed to find good parameters for py.SA, therefore, utilized it most often. When the FRAMAB's exploration component weight reached exploitation's, the other LLHs usage was triggered and as a result, HH-PC switched to j.ES usage. This switch gave a dramatic result improvements (code 2.4 \cref{eval:pict:hh-pc:pla7397 intermediate}). The FRAMAB LLH selection with BRR parameter control (code 2.5) at the beginning was using the mixture of mainly two j.ES and py.SA, but later switched to simulated annealing-only mode. As we may see, it gave a fast coarse-grained solution improvements in the beginning, and stable, but rather slow fine-grained improvements in a later stage.

On the final results quality charts (\cref{eval:pict:hh-pc:pla7397 final}) we observe a dominance of FRAMAB-based LLH selection (codes 2.4, 2.5) over BRR-based and TPE-based parameter control (codes 2.4 and 3.4) over BRR-based (codes 2.5, 3.5). On a contrary, the results obtained with BRR-based parameter control are more stable than TPE-based. The quality of final results did not reach the best performing tuned j.ES (code 4.3.2). However, since the optimization process did not settle in a local optima, we have a doubt that HH-PC will not outperform j.ES when given same number of external iterations.

For a better intuition, let us draw the reader's attention to \cref{eval:pict:hh-pc vs jES on pla7397 process}. Note how HH-PC (code 3.4) approaches j.ES progress curve. If it were not for the issue with a number of external iterations, the results would be better and, probably, outperforming tuned j.ES. Nevertheless, for current implementation and given 15 minutes for all solvers j.ES managed to make more iterations and moved further.

\begin{figure}[h]
	\centering
	\vspace{-5pt}
	\includesvg[width=\textwidth]{pla7397 HH-PC vs jES progress}
	\caption{HH-PC and tuned jMetal ES solving process comparison on pla7397 (statistic of 9 runs).}
	\vspace{-5pt}
	\label{eval:pict:hh-pc vs jES on pla7397 process}
\end{figure}

\paragraph{Discussion.} The observed results of solving the united ASP and PSP problems by HH-PC are encouraging. With small problem instances (kroA100, pr439, rat783) HH-PC managed to approach and in some cases even outperform the best underlying LLH with tuned hyper-parameters. When the problem size significantly grows (pla7397), the gap between HH-PC and the best performing LLH started to increase. An explanation for this behavior is simple: the system needs to build surrogate models not only for single, but for multiple LLHs and, therefore, requires more performance evidences in comparison to MH-PC or HH-SP. On a contrary, the amount of information only decreases as a consequence of an issue with py.ES. Since the parameter values are not selected to properly reflect the LLHs performance, the algorithm selection process also straggles. To overcome the problem with lack of information we propose to execute an additional meta-learning step before the beginning of optimization session. Unfortunately, we are forced to postpone an investigation of this rather intriguing idea for the future work. Nevertheless, HH-PC significantly exceeds the results of MHs with static default parameters in all cases and the guided by models approach outperforms fully randomized HH-PC.


\section{Parameter Analysis of Hyper-Heuristic with Parameter Control}\label{eval:2}
The second benchmark is dedicated to an evaluation of system settings influence on the solving process.
Due to limited time we decided to perform the benchmarks only for the most complex system mode: on-line selection hyper-heuristic with parameter control in low-level heuristics (HH-PC).

\subsection{Evaluation Plan}\label{eval:2:plan}
As with the concept evaluation, this set of benchmarks we also start from the planning. For comparison basis we selected one statistically better performing HH-PC setup, in which FRAMAB was used for the algorithm selection, and TPE for parameter control respectively. Their default configuration of which was as follows. Implemented FRAMAB algorithm exposes only one parameter: the balancing coefficient \emph{C}. In \cref{impl: FRAMAB} we discussed it values and also proposed an approach to replace this static value by one, derived from the deviation of results. In all previous tests we used exactly this approach. TPE implementation exposes \emph{split size} parameter, which defines the percentage of available data, used to create a distribution of good-performing parameter values (see TPE description during HpBandSter framework review in \cref{bg: bohb}). In all our experiments we used $\sfrac{1}{3}$ of available data to construct a `good' distribution. The amount of information, which was used to construct the surrogate models was controlled by a shared parameter \emph{window size}. In our evaluations we were using 80\% of available information each time. As one may remember, since we did not implement a proper optimization algorithm over surrogates, but rather used a random sampling and afterwards selected the best parameter set by means of their results with surrogates, we used a predefined number of randomly sampled parameter values on each level (more details in \cref{impl: prediction models}). During our experiments we sampled 96 combinations of parameter values on each level (the default value in BRISEv2). Also, in our setup we used a predefined number of workers (6 ps) and static time for task execution (15 seconds). The implemented RL approach required continuing the solving process between iterations. For doing so, after accomplishing a task the workers were sending a complete bunch of obtained solutions to main node, which is then attached them to the new tasks. This difference is caused by the usage of single-solution simulated annealing and population-based evolution strategy meta-heuristics.

All aforementioned characteristics form a three groups of experiments:
\begin{enumerate}
	\item \textbf{Learning granularity} group is designed to investigate the influence of performance evidences amount and quality, obtained between external iterations. It is formed by such parameters as \emph{window size}, \emph{task time} and \emph{number of workers}.
	
	\item \textbf{Learning models configuration} group of experiments are dedicated to investigate the influence of HLH parameters on the results quality and includes \emph{FRAMAB C coefficient}, \emph{TPE split size}, \emph{random search size} over the surrogates.
	
	\item \textbf{Generic low-level heuristic configuration} group is currently formed only from one characteristic for investigation: amount of solver warming-up information, which is defined by the number of solutions, reported in the end of LLH external iteration.
\end{enumerate}

We aggregate the proposed parameters and define the values for evaluation in \cref{eval:2: planning table}.

\begin{table}[h!]
	\centering
	\begin{tabular}{l||c|c}
		\textbf{Parameter} & \textbf{Investigated values} & \textbf{Default value} \\
		\hline
		\hline
		\rowcolor{gray!10}
		\multicolumn{3}{c}{Learning granularity} \\
		\textbf{Window size} & 30\%, 50\%, 100\% & 80\% \\
		\textbf{Task time} & 5, 10, 30 seconds & 15 seconds \\
		\textbf{Number of workers} & 3, 9, 12 & 6 \\
		\rowcolor{gray!10}
		\multicolumn{3}{c}{Learning models configuration} \\
		\textbf{FRAMAB C coefficient} & Static 0.001, 0.01, 0.1 & STD-based \\
		\textbf{TPE split size} & 10\%, 50\%, 70\% & 30\% \\
		\textbf{Random search size} & 50, 200 & 96 \\
		\rowcolor{gray!10}
		\multicolumn{3}{c}{Generic LLH configuration} \\
		\textbf{Warming-up solutions} & one & all
	\end{tabular}
	\caption{Prediction techniques used for the concept evaluation.}
	\label{eval:2: planning table}
\end{table}

We set all other parameters as they were configured during HH-PC evaluation, in particular the experiment running time is set to 15 minutes, number of experiment repetition is 9, the search space is unchanged (3 LLHs with the same parameter ranges and default values).

The idea of performing the full factorial design was quickly abandoned since it requires 46 days of non-stop experiment running.
Therefore, we performed a one-exchange benchmark design, which resulted in 18 experiments, which needed 40,5 hours to perform 9 repetitions.

\subsection{Learning Granularity}\label{eval:2:learning granularity}
In this experiment we investigate the influence of RL routines configuration on learning process. The idea is as follows. Changing the \emph{window size}, \emph{task time} and \emph{number of workers}, the underlying models may learn a different picture of dependencies. For instance, if we increase a task time, a sampled configuration will be measured more thoroughly, however, given a fixed experiment running time, the overall number of iteration will be decreased. On a contrary, by increasing a number of workers for the same fixed experiment time a size of investigated \emph{parameter space} will be increased, therefore, the underlying learning models will construct a more precise surrogates.

\textbf{Please note}, the perturbations at the end of runs on progress charts are caused by different number of performed iterations. More detailed explanation could be found in \cref{eval:1:baseline}.

\paragraph{Window size.}
\svgpath{{graphics/Eval/2bench}}
\begin{figure}[h]
	\centering
	\vspace{-20pt}
	\includesvg[width=\textwidth]{window size plots}
	\caption{Influence of HH-PC (code 2.4) window size on pla7397 TSP instance solving process.}
	\vspace{-5pt}
	\label{eval:2:pict:window size}
\end{figure}
According to our expectations in \cref{concept:prediction}, this parameter is responsible for the forgetting mechanism, which is required for system adaptation to changes in optimization process and domination of one parameters over others (including LLH).
In \cref{eval:2:pict:window size} we presented the system running results with four different window sizes. We observe that the best results were obtained, when the learning models used all available information, in other words, with disabled forgetting mechanism. On a contrary, with the smallest window size the quality of final results are the worst. A trend could be observed, according to which with a larger window size system provides a better quality of results. During the benchmarks, dedicated to the concept evaluation we used 80\% of available information, which corresponds to the middle-quality parameter value.
Our conclusion is as following: with this problem instance and system setup, the changes in a learning process and parameter preference are mostly negligible, therefore, the forgetting mechanism should be disabled.
In a future, it would be rather intriguing to investigate the influence of this parameter with other, potentially dynamic problem and/or instance, where the parameter preference is changing.

\paragraph{Task time.}
\begin{figure}[h]
	\centering
	\vspace{-20pt}
	\includesvg[width=\textwidth]{task time plots}
	\caption{Influence of HH-PC (code 2.4) task time on pla7397 TSP instance solving process.}
	\vspace{-5pt}
	\label{eval:2:pict:task time}
\end{figure}
Varying the time for one task evaluation (external iteration), one will dramatically change the granularity of obtained results. For instance, in our experiments with the least amount of time (5 seconds) HH-PC managed to perform more than 600 external iterations, while with the largest budged (30 seconds) approximately 150 (\cref{eval:2:pict:task time}). However, as we mentioned previously, the more information does not necessary imply a more precise surrogate models. The used by us during the concept evaluation 15 seconds provided in the worst results. Boundary cases with 5 and 30 seconds task time gave rather unstable result (see box-plots in \cref{eval:2:pict:task time}). The former is full of too approximately evaluated parameters, while the later simply did not manage to perform enough iterations. Balancing between results stability and quality we conclude that for the current setup statistically better choice would be to set 10 seconds for a task running time. 

\paragraph{Number of workers.}
\begin{figure}[h]
	\centering
	\vspace{-20pt}
	\includesvg[width=\textwidth]{number of workers plots}
	\caption{Influence of HH-PC (code 2.4) workers number on pla7397 TSP instance solving process.}
	\vspace{-5pt}
	\label{eval:2:pict:number of workers}
\end{figure}
The influence of workers number was rather surprise to us. According to our expectations, with a growing number of LLH runners, the amount of evaluated configurations increases. However, instead of getting surrogate models with a better quality and as a result, improving sampled parameters performance, we observe the opposite behavior. For increasing number of workers the results became worse and worse, see box-plots in \cref{eval:2:pict:number of workers}. Currently, we do not have a comprehensive explanation of the observed behavior, except of possible surrogate models over-fitting. It is broadly studied problem in ML, according to which the models create a too complex hypothesis, which can not generalize well to unforeseen data. Thus, in our case it is possible that being over-fit, models did not adequately predict a possible results for sampled parameters, therefore, guided prediction process in a wrong direction. In any case, this behavior requires more comprehensive investigation.

\subsection{Learning Models Configuration}\label{eval:2:learning models}
We perform this set of experiments for getting an intuition about the influence of underlying high-level heuristic configuration on general performance. As mentioned above, in HH-PC code 2.4 we used following learning models: FRAMAB for LLH selection and TPE for parameter tuning.

\paragraph{FRAMAB C coefficient.}
\begin{figure}[h]
	\centering
	\begin{subfigure}{\textwidth}
		\vspace{-10pt}
		\includesvg[width=\linewidth]{FRAMAB c plots}
		\caption{Statistical results comparison.}
		\label{eval:2:pict:framab c statistic}
	\end{subfigure}
	%\hfil 
	%\vspace{-5pt}
	\begin{subfigure}{\textwidth}
		\includesvg[width=\textwidth]{FRAMAB c plots chonics}
		\vspace{-5pt}
		\caption{One run with distinguished LLH allocation.}
		\label{eval:2:pict:framab c one run}
	\end{subfigure}
	\caption{Influence of HH-PC (code 2.4) FRAMAB C coefficient on pla7397 TSP instance solving process.}
	\label{eval:2:pict:framab c}
\end{figure}

As you remember from FRAMAB description, presented in \cref{impl: FRAMAB}, the role of coefficient C is to give the user possibility to control the EvE balance while selecting the LLH. In our implementation we proposed the usage of result improvement standard deviation, motivated by (1) the exploration encourage, while the uncertainty in results exist and (2) possibility of FRAMAB parameter-less usage. Therefore, in first set of benchmarks we exclusively used this, STD-based FRAMAB regime.

The results of performed benchmarks with possibly different C values are presented in \cref{eval:2:pict:framab c}. More concretely, in \cref{eval:2:pict:framab c statistic} we compare statistic of all repetitions with different parameter value. Here we may conclude, the proposed parameter-less (STD-based) FRAMAB performs slightly worse in comparison to other algorithms. Glancing on the sequence of LLH allocation, presented in \cref{eval:2:pict:framab c one run}, we conclude that parameter reflects its intent, especially, comparing $C=0.1$ and $C=0.001$. When C is large, the exploration-related part of FRAMAB's UCB value is increased, therefore, more different algorithms are used for optimization. However, when the C value is decreased, only few switches may be observed. An intriguing idea arises to introduce the technique for FRAMAB parameter control, according to which, the entire LLH portfolio utilized at the beginning with high C value, while approaching the end C is increased, to more concentrate with the best-performing LLH.

\paragraph{TPE split size.}
\begin{figure}[b]
	\centering
	\vspace{-10pt}
	\includesvg[width=\textwidth]{TPE split size plots}
	\caption{Influence of HH-PC (code 2.4) TPE split size on pla7397 TSP instance solving process.}
	\vspace{-5pt}
	\label{eval:2:pict:tpe split size}
\end{figure}
The internals of Bayesian TPE approach for parameter sampling were described during BOHB algorithm discussion in \cref{bg: bohb}. Evaluated in this experiment \emph{split size} parameter defines the proportion of information, used to construct the probability distributions of good parameter values. 

With small split size, only elite parameter values form distribution, therefore, an overall sampling process happens to be more greedy or, in other words, more exploitation-biased. According to our observations, presented in \cref{eval:2:pict:tpe split size}, the least greedy parameter allocation produces statistically the worst results (70 \%), while usage of 10 \% split relieved the best potential. Used in previous experiments 30 \% split produced slightly worse results than 50 \%, which still could be caused by the randomized processes. Probably, this behavior should be investigated using more statistical data.

\paragraph{Random search size.}
\begin{figure}[h]
	\centering
	\vspace{-10pt}
	\includesvg[width=\textwidth]{Random search size}
	\caption{Influence of HH-PC (code 2.4) random search size for surrogate optimization on pla7397 TSP instance solving process.}
	\vspace{-5pt}
	\label{eval:2:pict: random search size}
\end{figure}
This HLH parameter configures the random search process, performed over the created surrogate models. With given random search size equal $N$, for selecting the parameter values of LLH on each level $N$ randomized samples are taken. Afterwards, these samples are compared with each other using surrogate models and one with the best results became level prediction. Therefore, according to our expectations, by increasing the random search size, the parameter values prediction process become more precise and as a consequence, performance increases.

After evaluating two additional to default sizes of 50 and 200 samples respectively, the obtained results happen to be not as we expected. In general, we observe a quality fluctuation, which is caused by randomized processes. When the sampling size was decreased, the quality statistics was slightly improvement, which is non-logical behavior (see intermediate and final results in \cref{eval:2:pict: random search size}). It only motivates us to implement a proper optimization technique over surrogates for improving a robustness of the prediction process.

\subsection{Generic Low-Level Heuristics Configuration}\label{eval:2:llh changes}
\begin{figure}[h]
	\centering
	\vspace{-10pt}
	\includesvg[width=\textwidth]{warming-up solutions plots}
	\caption{Influence of HH-PC (code 2.4) warming-up solution number on pla7397 TSP instance solving process.}
	\vspace{-5pt}
	\label{eval:2:pict:warming-up solutions}
\end{figure}
The generic LLH configuration benchmark is represented by only one experiment, which we performed to evaluate how does the \emph{number of warming-up solutions} affects the search. Our intuition during the implementation was following: we have to initialize the solver with all previously available solutions not to lose the obtained trajectory and traversal velocity. It is relevant only to the population-based algorithms, such as evolution strategy or potential genetic algorithm. The implemented py.SA simply used the best one among available population to start. When SA completes its run, it reports only a single obtained solution. If the following LLH is population-based ES, we do not copy SA-based solution, but rather allow ES to sample all the rest uniformly at random.

However, after changing this behavior to the only one warming-up solution usage, we surprisingly found out that the results quality was almost not affected: please, pay attention to overlapping progress curves in a left side of \cref{eval:2:pict:warming-up solutions}. Moreover, passing only one solution between LLHs the overall number of external iterations were increased dramatically (from ~200 to ~300). It is caused by the reduced overhead for information processing and sending through network. We conclude that changing the behavior to only one warming-up solution usage provides a positive impact on the final results quality and do not affect the intermediate progress.



\section{Conclusion}\label{eval: conclution}
The evaluation of proposed concept was presented in this chapter and performed in two stages. At the first stage an analysis of the implemented concept was performed. We compared it with a baseline, defined by the executed in isolation underlying meta-heuristics with static hyper-parameters. Our conclusions on the concept applicability are following:
\begin{itemize}
	\item Firstly, the proposed reinforcement learning-based generic parameter control approach (MH-PC) is able to significantly improve the performance of meta-heuristic's static hyper-parameters and in some cases even outperform the results of tuned beforehand parameters. From the implementation perspective, the system has lack of mechanism for bad-performing external iteration termination. On the generic level, it may be implemented in form of tasks termination mechanism, carried out by BRISEv2.
	
	\item Secondly, the developed heuristic selection technique (HH-SP) is able to reach a quality of the best performing underling low-level heuristic. 
	
	\item Finally, the approach for simultaneous algorithm selection and parameter tuning (HH-PC) in runtime outperforms the best underlying algorithm with tuned parameters on rather small problem instances (kroA100, pr439). With growing complexity, the aforementioned issue of LLH struggling results in a gap between the desired performance of the best tuned algorithm and our approach. But, the guided by RL and surrogate models HH-PC considerably outperforms randomized selection of both LLH and parameter, which is the only approach available in runtime. A more confident conclusion for larger problem instances should be done after thorough investigation, where low-level heuristics are given more time to converge in local optimum and aforementioned issues are resolved.
\end{itemize}

In the second evaluation part we investigated an influence of the proposed united approach (HH-PC) configuration setting on its performance. The influence of some among evaluated parameters was not as we expected, which only motivates the importance of proper parameter values search and usage. More concretely, an insertion of relatively `light' \emph{forgetting mechanism} instead of improving an optimization process made the results statistically sightly worse. The increased number of workers only decreased the surrogate models accuracy. The decision of all available solutions usage for LLH initialization introduced a redundant overhead, but did not improve final results quality, therefore, it should be reconsidered. Nevertheless, a set of experiments from the second evaluation stage should also be performed over the other system operating modes (MH-PC and HH-SP) for making a confident conclusion. Up until now it reviled the system adaptation ability with help of exposed parameters.
\todoy{add evaluation of 2.4 on a largest problem instance will all reviewed fixes?}
  \chapter{Conclusion}\label{conclusion}
Before making a final conclusion, let us briefly remind our objective.
The task of this thesis was defined as follows: using an existing parameter tuning software proposes a concept to (1) perform the parameter control in meta-heuristics on a generic level and (2) make both algorithm selection and parameter control solve the optimization problem at hand. In our research the complex objective was split into several compound tasks, which were formulated in three research questions (\cref{intro: research objective}). Here we provide explicit answers to each of them.

The proposed in this thesis generic parameter control approach relies on two aspects. Firstly, it should be possible to evaluate the performance of the system under control with specified configuration at any time. The ideal option is to limit the system execution with the budget specified beforehand (number of iterations, wall-clock time, etc). Secondly, it should be possible to change the target algorithm configuration and proceed with the execution basing on previously obtained results. We use the reinforcement learning methodologies to traverse the parameter space, evaluating the performance of unforeseen configurations iteratively, while solving the problem at hand. The proposed concept of generic parameter control was examined in \cref{eval:1:PC} with three meta-heuristics: two Python-based algorithms, namely, simulated annealing and evolution strategy and one Java-based evolution strategy. The proposed approach revealed its applicability by reaching, and in some cases even outperforming the results of tuned in offline algorithm parameters. Therefore, we answer the \textbf{RQ1}: it is indeed possible to perform the algorithm configuration at runtime on the generic level. Please note, the use-cases of our concept are defined by the algorithms, in which execution time is much larger than time spent to parameter control routines (see \cref{concept:parameter control}).

\paragraph{RQ2} \emph{Is it possible to simultaneously perform algorithm selection and parameters adaptation while solving an optimization problem?}

Our idea of merging those two problems lays in treating the algorithm type as a regular categorical parameter in the search space. By utilizing the parent-child relationships we define dependent parameters in such search space and perform the selection by means of firstly sampling the independent parameters and hiding the children, and secondly fixing the selected for parents' values and exposing the \emph{activated} children parameters. The proposed stepwise process of configuration construction provides a possibility to utilize a wide range of surrogate models for learning the dependencies among parameter values on each level in isolation. The requirements and use-cases of the proposed approach remain the same as for the generic parameter control technique defined above.


\paragraph{RQ3} \emph{What is the effect of selecting and adapting algorithms while solving an optimization problem?}

The performed in \cref{eval:1:hh-pc} evaluation and analysis of the simultaneous online algorithm selection and parameter control revealed its applicability. More specifically, given the same wall-clock time and environment setup, HH-PC was able to outperform the best underlying tuned meta-heuristics for kroA100 TSP instance. For a slightly larger pr439 example, HH-PC results were comparable with the best tuned in offline meta-heuristic. With problem size growing further, the gap between HH-PC and the best available solver, used in isolation with tuned parameters increases. However, comparing the averaged quality of all available solvers used in isolation with the results of our approach, we observe a strong domination of later. Therefore, our answer to \textbf{RQ3} is the following: HH-PC constantly produces good quality results and should be considered if the meta-heuristics domination and their hyper-parameters are unknown beforehand. However, a considerable amount of solution quality is sacrificed to tackle APSP problem, in comparison to the best underlying algorithm usage with tuned parameters.


\paragraph{}
An explicit list of this thesis contributions is the following:
\begin{enumerate}
	\item The concept of reinforcement learning-based generic parameter control in meta-heuristics was proposed and empirically evaluated.% on three meta-heuristic implementations proved its ability to reach the performance of tuned in offline parameters.
	
	\item The unification of both online algorithm selection and generic parameter control approaches into single APSP is performed. The approach to tackle APSP is proposed by means of reinforcement learning-based online selection hyper-heuristic with parameter control in low-level meta-heuristics.

	\item The usability of an existing parameter tuning SPL BRISEv2 was extended with aforementioned use-cases without losing the flexibility of a wide range of learning models usage. Moreover, the concept of data preprocessing was encapsulated.
\end{enumerate}

We consider the task of this thesis accomplished and the proposed reinforcement learning-based generic parameter control and algorithm selection approaches useful for solving the optimization problems with the help of meta-heuristics.
  \chapter{Future work}
In this Chapter we discuss the investigations that we postponed as a future work. They could be grouped in the several sets therefore, we discuss them in a separate sections. Thus, in the \cref{fw: search space} we discuss a set of enhancements for the search space implementation that should be performed to better suite for users needs. The \cref{fw: prediction process} is dedicated to the prediction process and learning models, which guide the system performance. Finally, in the section \cref{fw: evaluation} we discuss a benchmark experiments to obtain a better evidences about the implemented solution applicability.

\section{Search Space}\label{fw: search space}
\paragraph{Composition on numerical parameters} In the \cref{impl: search space impl} we highlighted that the implemented search space is able to form the parent-child relationship only when parent is of the categorical type. However, it may happen, when the dependencies among parameters are based on their numeric values. As instance, for one range the first child type should be exposed, while for other range — another child. As a hint for future implementation we propose utilizing a similar to used in categorical parameters approach however, instead of activation values, use the ranges of values. Thus, during the prediction propagation step the parameter will simply check all ranges and trigger the related children (for more details see description of the \cref{impl: P.F.R.1 + P.F.R.3 implementation pseudocode}).

\paragraph{Constraints among parameters} Sometimes, the prohibitions for a specific values may arise with respect to other parameters. For instance, the value of one numeric parameter should be at least as high as value of the other. 


\section{Prediction Process}\label{fw: prediction process}
\paragraph{add more sophisticated models}
\paragraph{technique to optimize obtained surrogate model should be generalized}
I did not investigate decoupling the surrogate models from the search algorithm to optimize those surrogates (done in Sasha's thesis).
\paragraph{investigate more deeply decoupling of data preprocessing and learning algorithm} auto-sklearn
\paragraph{influence for warm-start onto this kind of HH (by off-line learning)}
Although, the influence of meta-learning, applied in Auto-Sklearn system~\cite{feurer2015efficient} to warm-start the learning mechanism proved to worth the effort spent, as well as it was reported by developers of Selective Hyper-Heuristics with mixed type of learning~\cite{uludaug2013hybrid,}, it is intriguing to check an influence of metal-learning onto Selective Hyper-Heuristics with Parameter Control.
Also, $https://ml.informatik.uni-freiburg.de/papers/20-ECAI-DAC.pdf$
\paragraph{Random Forest HLH surrogate model}
% TODO: try a random forest as a 1ST level of HH, SMAC paper (short version), PAGE 7, CITES 18, 19.
\paragraph{add other learning metrics}
Inspiration could be found at:
- EAs in~\cite{karafotias2014generic}: progress stagnation, 


\section{Evaluations and Benchmarks}\label{fw: evaluation}
\paragraph{add new class of problem (jmetalpy easly allows it)}
\paragraph{evaluation on different types and classes}
\paragraph{interesting direction: apply to automatic machine learning problems solving, compare to auto-sklearn.}
\paragraph{bounding LLH by number of evaluations, not time}
\paragraph{adaptive time for tasks}


  
  \printbibliography[heading=bibintoc]\label{sec:bibliography}
  
  \listoffigures
  \listoftables

  \appendix
  \chapter{Evaluation plots}
\section{Concept Evaluation}

\subsection{Baseline}\label{app:eval:bl plots}
\svgpath{{graphics/Eval/baseline}}

\begin{figure}[h]
	\centering
	\includesvg[width=\textwidth]{kroA100 baseline progress}
	\caption{Intermediate results of meta-heuristics with static parameters on kroA100.}
	\label{app:eval:1:bl:kroA100 intermediate}
\end{figure}

\begin{figure}[h]
	\centering
	\includesvg[width=\textwidth]{pr439 baseline progress}
	\caption{Intermediate results of meta-heuristics with static parameters on pr439.}
	\label{app:eval:1:bl:pr439 intermediate}
\end{figure}


\begin{figure}[h]
	\centering
	\begin{subfigure}{0.45\textwidth}
		\includesvg[width=\textwidth]{kroA100 baseline final boxplot}
		\caption{kroA100 TSP instance.}
		\label{app:eval:1:bl:kroA100 final}
	\end{subfigure}
	\begin{subfigure}{0.45\textwidth}
		\includesvg[width=\linewidth]{pr439 baseline final boxplot}
		\caption{pr439 TSP instance.}
		\label{app:eval:1:bl:pr439 final}
	\end{subfigure}
	\caption{Final results of meta-heuristics with static parameters.}
	
\end{figure}


\subsection{Parameter Control}\label{app:eval:pc plots}
\svgpath{{graphics/Eval/control}}
\begin{figure}[h]
	\centering
	\includesvg[width=\textwidth]{pr439 PC progress}
	\caption{Intermediate results of meta-heuristics with parameter control on pr439.}
	\label{app:eval:1:pc:pr439 intermediate}
\end{figure}

\begin{figure}[h]
	\centering
	\includesvg[width=\textwidth]{pr439 PC final boxplot}
	\caption{Final results of meta-heuristics with parameter control on pr439.}
	\label{app:eval:1:pc:pr439 final}
\end{figure}


\subsection{Selection Hyper-Heuristic with Static LLH Parameters}\label{app:eval:hh-sp}
\svgpath{{graphics/Eval/selection}}
\begin{figure}[h]
	\centering
	\includesvg[width=\textwidth]{kroA100 HH-SP progress}
	\caption{Intermediate performance of on-line selection hyper-heuristic with static hyper-parameters on kroA100 (single experiment).}
	\label{app:eval:1:hh-sp:kroA100 intermediate}
\end{figure}

\begin{figure}[h]
	\centering
	\includesvg[width=\textwidth]{kroA100 HH-SP final boxplot}
	\caption{Final results of on-line selection hyper-heuristic with static hyper-parameters on kroA100 (statistic of 9 runs).}
	\label{app:eval:1:hh-sp:kroA100 final}
\end{figure}

\begin{figure}[h]
	\centering
	\includesvg[width=\textwidth]{pr439 HH-SP progress}
	\caption{Intermediate performance of on-line selection hyper-heuristic with static hyper-parameters on pr439 (single experiment).}
	\label{app:eval:1:hh-sp:pr439 intermediate}
\end{figure}

\begin{figure}[h]
	\centering
	\includesvg[width=\textwidth]{pr439 HH-SP final boxplot}
	\caption{Final results of on-line selection hyper-heuristic with static hyper-parameters on pr439 (statistic of 9 runs).}
	\label{app:eval:1:hh-sp:pr439 final}
\end{figure}

\subsection{Selection Hyper-Heuristic with Parameter Control}\label{app:eval:hh-pc}
\svgpath{{graphics/Eval/hhpc}}
\begin{figure}[t]
	\centering
	\includesvg[width=\textwidth]{kroA100 HH-PC progress}
	\caption{Intermediate performance of HH-PC on kroA100 (single experiment).}
	\label{app:eval:1:hh-pc:kroA100 intermediate}
\end{figure}

\begin{figure}[b]
	\centering
	\includesvg[width=\textwidth]{kroA100 HH-PC final boxplot}
	\caption{Final results of HH-PC compared with MH on kroA100 (statistic of 9 runs).}
	\label{app:eval:1:hh-pc:kroA100 final}
\end{figure}

\begin{figure}[t]
	\centering
	\includesvg[width=\textwidth]{pr439 HH-PC progress}
	\caption{Intermediate performance of HH-PC on pr439 (single experiment).}
	\label{app:eval:1:hh-pc:pr439 intermediate}
\end{figure}

\begin{figure}[b]
	\centering
	\includesvg[width=\textwidth]{pr439 HH-PC final boxplot}
	\caption{Final results of HH-PC compared with MH on pr439 (statistic of 9 runs).}
	\label{app:eval:1:hh-pc:pr439 final}
\end{figure}


  %\newpage
  %\pagestyle{empty}
  %\section*{Confirmation}
  %	I confirm that I independently prepared the thesis and that I used only the references and auxiliary means indicated in the
  %	thesis.
  %	\\
  %	\\
  %	\\
  %	\\
  %	\\
  %	\\
  %	\\
  %	Dresden, 11th May 2020
  
\end{document}
